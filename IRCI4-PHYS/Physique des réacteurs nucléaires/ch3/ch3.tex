\chapter{Approximation de l'équation de diffusion}
\section{Équation de Boltzmann monocinétique}
\subsection{Équation de transport monocinétique}
Reprenons notre fameuse équation
\begin{equation}
\begin{array}{ll}
\DS\bar \Omega .\bar \nabla \varphi (\bar r,v,\bar \Omega ) &\DS+ {\Sigma _t}(\bar r,v)\varphi (\bar r,v,\bar \Omega ) - \int\limits_{4\pi }    \int_o^\infty     {\Sigma _s}(\bar r,v',\bar \Omega ' \to v,\bar \Omega )\varphi (\bar r,v',\bar \Omega ')dv'd\bar \Omega '\\
&\DS = \frac{1}{{4\pi }}\chi (v)\int\limits_{4\pi }    \int_o^\infty     \nu {\Sigma _f}(\bar r,v')\varphi (\bar r,v',\bar \Omega ')dv'd\bar \Omega ' + Q(\bar r,v,\bar \Omega )
\end{array}
\end{equation}
Supprimons la dépendance en vitesse dans l'équation de Boltzmann
\begin{equation}
\bar \Omega .\bar \nabla \varphi (\bar r,\bar \Omega ) + {\Sigma _t}(\bar r)\varphi (\bar r,\bar \Omega ) - \int\limits_{4\pi }   {\Sigma _s}(\bar r,\bar \Omega ' \to \bar \Omega )\varphi (\bar r,\bar \Omega ')d\bar \Omega '=  = \frac{{\nu {\Sigma _f}(\bar r)}}{{4\pi }}\int\limits_{4\pi }    \varphi (\bar r,\bar \Omega ')d\bar \Omega ' + Q(\bar r,\bar \Omega )
\end{equation}
Le terme de scattering ne fait plus que changer la direction de propagation et non plus la vitesse. Dès
lors, la probabilité que les neutrons soit émis à vitesse unique vaut forcément l'unité. \\

Introduisons le n ombre de neutrons secondaires par interaction
\begin{equation}
c(\bar r) = \frac{{{\Sigma _s}(\bar r) + \nu {\Sigma _f}(\bar r)}}{{{\Sigma _t}(\bar r)}}
\end{equation}
Nous avons déjà pris en considération que la probabilité de capture d'un neutron secondaire est 
nulle, pour une fission également et pour le scattering, l'unité\footnote{Incohérence.}. En effet, 
la valeur moyenne c'est 0*proba1+1*porba2+1*proba3???\\

Considérons la distribution angulaire de scattering
\begin{equation}
\frac{1}{{2\pi }}f(\bar r,\bar \Omega .\bar \Omega ') = \frac{1}{{c(\bar r){\Sigma _t}(\bar r)}}({\Sigma _s}(\bar r,\bar \Omega ' \to \bar \Omega ) + \frac{{\nu {\Sigma _f}(\bar r)}}{{4\pi }})
\end{equation}
En regroupant les termes e l'équation ci-dessus ensemble, on va avoir dans une même intégrale la 
section différentielle et la $\nu\Sigma_f/4\pi$. Si l'on fait l'intégrale de cette densité de 
probabilité sur tous les $\bar\Omega$ on obtient bien 1.  Après ré-écriture
\begin{equation}
\bar \Omega .\bar \nabla \varphi (\bar r,\bar \Omega ) + {\Sigma _t}(\bar r)\varphi (\bar r,\bar \Omega ) - \frac{{c{\Sigma _t}(\bar r)}}{{2\pi }}\int\limits_{4\pi }    f(\bar \Omega '.\bar \Omega )\varphi (\bar r,\bar \Omega ')d\bar \Omega ' = Q(\bar r,\bar \Omega )
\end{equation}
Dans le membre de gauche, on a introduit pour le scattering la section différentielle. Pour des 
raisons de symétrie (azimutale) il n'y a pas de raison qu'un angle $\phi$ apparaisse comme 
direction de scattering : on peut clairement mettre en évidence $1/2\pi$ en supposant que l'on 
travaille avec les $\cos\theta$ des coordonnées classiques pour obtenir cette forme compacte.\\

Les fonction de Legendre sont pratiques lorsqu'un terme dépend de $\theta$ ou $\cos\theta$
\begin{equation}
f(\mu ) = \sum\limits_l    \frac{{2l + 1}}{2}{f_l}{P_l}(\mu ),\qquad \mu  = \bar \Omega .\bar \Omega ',\qquad {P_l}(\mu ) = \sum\limits_{m = 0}^{l/2}    {( - 1)^m}\frac{{(2l - 2m)!}}{{{2^l}m!(l - m)!(l - 2m)!}}{\mu ^{l - 2m}},
\end{equation}
\begin{equation}
\int_{ - 1}^{ + 1}    {P_m}(\mu ){P_n}(\mu )d\mu  = \frac{{2{\delta _{mn}}}}{{2n + 1}}, \qquad
{f_l} = \int_{ - 1}^{ + 1} {  {P_l}(\mu )f(\mu )d\mu } 
\end{equation}
Si l'anisotropie est faible, on va se limiter au premier ordre dans la dépendance en les angles. Le 
scattering de fission peut alors se ramener à une expression qui fait apparaitre trois fois la 
moyenne de $\cos\theta$, on y reviendra. L'expression devient alors\footnote{Il faut savoir justifier?}
\begin{equation}
\bar \Omega .\bar \nabla \varphi (\bar r,\bar \Omega ) + {\Sigma _t}(\bar r)\varphi (\bar r,\bar \Omega ) - \frac{{c{\Sigma _t}(\bar r)}}{{4\pi }}\int\limits_{4\pi }    (1 + 3 < {\mu _o} > \bar \Omega '.\bar \Omega )\varphi (\bar r,\bar \Omega ')d\bar \Omega ' = Q(\bar r,\bar \Omega )
\end{equation}


\subsection{Forme intégrale}
Dans le cas d'un scattering isotropique et d'une source, nous avions vu au chapitre précédent que 
le flux s'exprimait
\begin{equation}
\varphi (\bar r,v) = \int_{{R^3}}^{} {\frac{{{e^{ - {\tau _v}(\bar r,{{\bar r}_o})}}}}{{4\pi {{\left| {\bar r - {{\bar r}_o}} \right|}^2}}}} S({\bar r_o},v)d{\bar r_o}
\label{eq:Ch3.cp}
\end{equation}
Dans le cas monocinétique, 
\begin{equation}
\varphi (\bar r) = \int_{{R^3}}^{} {  \frac{{{e^{ - {\tau _v}(\bar r,{{\bar r}_o})}}}}{{4\pi {{\left| {\bar r - {{\bar r}_o}} \right|}^2}}}} (c{\Sigma _t}({\bar r_o})\varphi ({\bar r_o}) + Q({\bar r_o}))d{\bar r_o}
\end{equation}
où on définit le \textbf{noyau de transport}
\begin{equation}
K(\bar r,{\bar r_o}) = \frac{{{e^{ - {\tau _v}(\bar r,{{\bar r}_o})}}}}{{4\pi {{\left| {\bar r - {{\bar r}_o}} \right|}^2}}}
\end{equation}
Tout ceci signifie que pour une source ponctuelle en $\bar r_0$, le flux que l'on va avoir dans le 
milieu est purement absorbant à cause de l'exponentielle négative dans le noyau ; pas de 
multiplication des neutrons et $C=0$ si purement absorbant. On trouvera alors comme solution pour 
un point source dans un milieu purement absorbant
\begin{equation}
Q(\bar r,\bar \Omega ) = \frac{1}{{4\pi }}\delta (\bar r - {\bar r_o})
\end{equation}

\newpage
\subsection{Théorème de réciprocité et théorème}
\theor{
\begin{equation}
\varphi (\bar r,\bar \Omega |{\bar r_o},{\bar \Omega _o}) = \varphi ({\bar r_o}, - {\bar \Omega _o}|\bar r, - \bar \Omega )
\end{equation}
où $\DS\bar \Omega .\bar \nabla \varphi (\bar r,\bar \Omega ) + {\Sigma _t}(\bar r)\varphi (\bar r,\bar \Omega ) - \frac{{c{\Sigma _t}(\bar r)}}{{2\pi }}\int\limits_{4\pi }  f(\bar \Omega '.\bar \Omega )\varphi (\bar r,\bar \Omega ')d\bar \Omega ' = Q(\bar r,\bar \Omega )$.}\ \\

Il s'agit du théorème équivalent au "renversement du temps" en mécanique rationnelle.\\

\begin{proof}\ \\
Nous aimerions prouver que le flux en un point $\bar r$ dans la direction $\bar \Omega$ à partir 
d'une source ponctuelle en $\bar r_0$ et monodirectionnelle en $\bar r_0$ est équivalent au flux 
en $\bar r_0$ dans la direction de propagation $-\Omega r_0$ pour une source en $\bar r$ de 
direction $-\bar\Omega$. Pour se faire, écrivons l'équation de transport en faisant apparaître les 
flux étant donné la source ponctuelle en une direction. La source extérieure est le membre de 
droite (neutrons émis en $\bar r_0$ dans une direction $\bar \Omega_0$).
\begin{equation}
\bar \Omega .\bar \nabla \varphi (\bar r,\bar \Omega |{{\bar r}_o},{{\bar \Omega }_o}) + {\Sigma _t}(\bar r)\varphi (\bar r,\bar \Omega |{{\bar r}_o},{{\bar \Omega }_o}) - \frac{{c(\bar r){\Sigma _t}(\bar r)}}{{2\pi }}\int\limits_{4\pi }    f(\bar \Omega '.\bar \Omega )\varphi (\bar r,\bar \Omega '|{{\bar r}_o},{{\bar \Omega }_o})d\bar \Omega ' = \delta (\bar r - {{\bar r}_o})\delta (\bar \Omega  - {{\bar \Omega }_o})
\label{eq:Ch3.1}
\end{equation}
Écrivons cette même équation mais pour un flux au point $\bar r$ dans la direction $-\bar\Omega$ 
pour une source de direction $-\bar\Omega_1$ (le fait d'avoir changé $\bar \Omega$ en $-\bar \Omega$
ne change rien car tous les cas de figures seront pris en compte lors de l'intégration
\begin{equation}
 - \bar \Omega .\bar \nabla \varphi (\bar r, - \bar \Omega |{{\bar r}_1}, - {{\bar \Omega }_1}) + {\Sigma _t}(\bar r)\varphi (\bar r, - \bar \Omega |{{\bar r}_1}, - {{\bar \Omega }_1}) - \frac{{c(\bar r){\Sigma _t}(\bar r)}}{{2\pi }}\int\limits_{4\pi }    f(\bar \Omega '.\bar \Omega )\varphi (\bar r, - \bar \Omega '|{{\bar r}_1}, - {{\bar \Omega }_1})d\bar \Omega '
\quad \quad \quad \quad \quad \quad \quad \quad \quad \quad  = \delta (\bar r - {{\bar r}_1})\delta ( - \bar \Omega  + {{\bar \Omega }_1})
\label{eq:Ch3.2}
\end{equation}
Multiplions \eqref{eq:Ch3.1} par le flux de la seconde ($\times \varphi (\bar r, - \bar \Omega |{\bar r_1}, - {\bar \Omega _1})$) et inversement pour \eqref{eq:Ch3.2} ($ \times \varphi (\bar r,\bar \Omega |{\bar r_o},{\bar \Omega _o})$). En effectuant la différence
\begin{equation}
\begin{array}{l}
\bar \Omega .\bar \nabla (\varphi (\bar r,\bar \Omega |{{\bar r}_o},{{\bar \Omega }_o}).\varphi (\bar r, - \bar \Omega |{{\bar r}_1}, - {{\bar \Omega }_1})) - \frac{{c(\bar r){\Sigma _t}(\bar r)}}{{2\pi }}\int\limits_{4\pi }    f(\bar \Omega '.\bar \Omega )[\varphi (\bar r, - \bar \Omega |{{\bar r}_1}, - {{\bar \Omega }_1})\varphi (\bar r,\bar \Omega '|{{\bar r}_o},{{\bar \Omega }_o})\\
\quad \quad \quad \quad \quad \quad \quad \quad \quad \quad \quad \quad \quad \quad \quad \quad \quad \quad \quad \quad \quad \quad \quad \quad \quad  - \varphi (\bar r,\bar \Omega |{{\bar r}_o},{{\bar \Omega }_o})\varphi (\bar r, - \bar \Omega '|{{\bar r}_1}, - {{\bar \Omega }_1})]d\bar \Omega '\\
\quad \quad \quad \quad \quad \quad \quad \quad \quad \quad  = \varphi (\bar r, - \bar \Omega |{{\bar r}_1}, - {{\bar \Omega }_1})\delta (\bar r - {{\bar r}_o})\delta (\bar \Omega  - {{\bar \Omega }_o}) - \varphi (\bar r,\bar \Omega |{{\bar r}_o},{{\bar \Omega }_o})\delta (\bar r - {{\bar r}_1})\delta (\bar \Omega  - {{\bar \Omega }_1})
\end{array}
\label{eq:Ch3.3}
\end{equation}
Intégrons sur le volume du réacteur
\begin{equation}
\int_V \eqref{eq:Ch3.3}\ \text{d}\bar r
\end{equation}
Les $\delta_{DiDi}$ du membres de droite sautent. L'intégrale du $\bar\Omega\bar\nabla$ va redonner
une intégrale de la divergence de $\bar\Omega\varphi_0\varphi_1$ soit une intégrale de surface du 
produit des deux flux avec un vecteur normal orientée vers l'extérieur. Comme on intègre sur toute 
la surface du réacteur nous aurons d'un côté $\bar\Omega$ et de l'autre $-\bar\Omega$ : il y a 
toujours un facteur qui correspond à un flux entrant, nous permettant d'appliquer les conditions 
aux limites du vide
\begin{equation}
\int_V \bar\Omega.\bar \nabla(\varphi_0,\varphi_1)d\bar r = \int_V \div \bar\Omega \varphi_0
\varphi_1)d\bar r,\qquad \oint_S = \varphi_0\varphi_1 \bar\Omega ds
\end{equation}
Comme on considère ce produit pour tous les $\bar r$ de la surface, ce qui est $\bar \Omega$ sortant
pour le premier facteur est sortant pour le second : grace à ceci, nous avons
\begin{equation}
\begin{array}{l}
\bar \Omega .\bar \nabla (\varphi (\bar r,\bar \Omega |{{\bar r}_o},{{\bar \Omega }_o}).\varphi (\bar r, - \bar \Omega |{{\bar r}_1}, - {{\bar \Omega }_1})) - \frac{{c(\bar r){\Sigma _t}(\bar r)}}{{2\pi }}\int\limits_{4\pi }   f(\bar \Omega '.\bar \Omega )[\varphi (\bar r, - \bar \Omega |{{\bar r}_1}, - {{\bar \Omega }_1})\varphi (\bar r,\bar \Omega '|{{\bar r}_o},{{\bar \Omega }_o})\\
\quad \quad \quad \quad \quad \quad \quad \quad \quad \quad \quad \quad \quad \quad \quad \quad \quad \quad \quad \quad \quad \quad \quad \quad \quad  - \varphi (\bar r,\bar \Omega |{{\bar r}_o},{{\bar \Omega }_o})\varphi (\bar r, - \bar \Omega '|{{\bar r}_1}, - {{\bar \Omega }_1})]d\bar \Omega '\\
\quad \quad \quad \quad \quad \quad \quad \quad \quad \quad  = \varphi (\bar r, - \bar \Omega |{{\bar r}_1}, - {{\bar \Omega }_1})\delta (\bar r - {{\bar r}_o})\delta (\bar \Omega  - {{\bar \Omega }_o}) - \varphi (\bar r,\bar \Omega |{{\bar r}_o},{{\bar \Omega }_o})\delta (\bar r - {{\bar r}_1})\delta (\bar \Omega  - {{\bar \Omega }_1})
\end{array}
\label{eq:Ch3.4}
\end{equation}
Intégrons ceci sur toutes les directions
\begin{equation}
\int_{4\pi} \eqref{eq:Ch3.4}\ \text{d}\bar\Omega
\end{equation}
Grâce aux $\delta$, si on parvient à montrer que le membre de gauche est nul (en oubliant les indices
1) nous aurons démontrer la proposition. Est-ce le cas ? Nous avons $f(\bar \Omega'\bar \Omega)$. Si 
on pose $\bar \Omega=\bar \Omega'$ (simplement changement de variable) cela n'a aucune conséquence, 
c'est totalement équivalent. Mais nous avons alors les deux mêmes termes qui se soustraient, la 
proposition est démontrée.
\end{proof}
Si la source est isotrope, en intégrant le résultat précédent on en tire que le flux en un point 
$\bar r$ pour une source $\bar r_0$ sera le même que le flux en $\bar r_0$ pour une source en 
$\bar r$.
\begin{equation}
\varphi (\bar r|{\bar r_o}) = \varphi ({\bar r_o}|\bar r)
\end{equation}
Il s'agit d'un corollaire de notre théorème. 

\subsubsection{Probabilité de collision}
Soit un ensemble de zones homogènes $V_i$. On définit $P_{i\to j}^t$ comme la probabilité qu'un 
neutron qui est émis de façon uniforme et isotrope dans $V_i$ va faire sa prochaine collision dans 
$V_j$.
\begin{equation}
P_{i \to j}^t = \int\limits_{{V_i}}   \int\limits_{{V_j}} {  {\Sigma _{tj}}} \varphi (\bar r|{\bar r_o})\frac{1}{{{V_i}}}d\bar rd{\bar r_o}
\end{equation}
où $1/V_id\bar r_0$ est le nombre de neutrons émis en $d\bar r_0$ autour de $\bar r_0$ et 
$\Sigma_{tj}\varphi(\bar r|\bar{r_0})d\bar{r}$ est le taux de réaction dans $d\bar{r}$ autour de
$\bar r$ par neutrons émis en $\bar{r_0}$.Il s'agit donc de l'intégration d'un taux de réaction sur
$V_j$ avec un $d\bar r$ qui correspond aux variables du volume cible, il s'agit bien d'une 
probabilité.\\

Si une source donne un neutron, ce volume intégré sur $V_j$ va donner la fraction par rapport à ce 
neutron qui va avoir une interaction. Quel est l'intensité d'une telle source pour être uniforme ? Il 
faut intégrer sur $V_i$ et diviser par $V_i$. La source d'intensité $1/V_i$ élément pendant une 
introduit un temps implicite qui compense le "par unité de temps" que l'on retrouve dans le 
$\Sigma_t\varphi$. L'intensité de la source uniforme est donc $1/V_i$ et on peut dire que 
$\Sigma_t$ est constant sur un volume $V_j$.\\

Nous avons alors
\begin{equation}
\frac{{{V_i}P_{i \to j}^t}}{{{\Sigma _{tj}}}} = \int\limits_{{V_i}}   \int\limits_{{V_j}}   \varphi ({\bar r_o}|\bar r)d\bar rd{\bar r_o} = \frac{{{V_j}P_{j \to i}^t}}{{{\Sigma _{ti}}}}
\end{equation}
où nous avons appliqué le théorème de réciprocité en permutant $i$ et $j$. En en tire que
\begin{equation}
{\Sigma _{ti}}{V_i}P_{i \to j}^t = {\Sigma _{tj}}{V_j}P_{j \to i}^t
\end{equation}
Il s'agit d'une autre forme directe du théorème de réciprocité.

\subsubsection{Probabilité de fuite}
Considérons une région $V$ homogène de surface $S$. La probabilité de fuite signifie que l'on va 
devoir considérer un courant ($\varphi.\bar\Omega$) par neutron émis (plutôt qu'un taux de réaction)
d'où la variation autour de la direction. Soit $P_0$ la probabilité de fuite pour un neutron apparaissant uniformément et de façon isotrope dans $V$
\begin{equation}
{P_o} = \int\limits_S    \int\limits_{\bar n.\bar \Omega  > 0}    \int\limits_V   \int\limits_{{{\bar \Omega }_o}}    \varphi ({\bar r_s},\bar \Omega |{\bar r_o},{\bar \Omega _o})\frac{1}{{4\pi V}}d{\bar r_o}d{\bar \Omega _o}\bar n.\bar \Omega d\bar \Omega dS
\end{equation}
Pour avoir le courant en tout point de la surface en direction $\bar\Omega$ étant donné une source
en $\bar r_0$ de direction $\bar\Omega_0$ il faut intégrer le courant sur toutes les directions 
sortantes soit sur toute la surface avec la normale sortante. Il s'agit de la partie d'intégration 
$\bar r_s$ sur ces variables $\bar\Omega$. Comme il faut tenir compte de la position et de la 
direction de propagation, on retrouve $1/V$.\footnote{Pq?} Le $1/4\pi$ est la pour que ce soit isotrope. L'intégration de $1/4\pi V$ donne bien 1, soit le nombre de neutrons sources. Rappelons que 
l'intégration se fait sur la normale sortante car on s'intéresse au courant de fuite.\\

On définit $\Gamma_0$ comme la probabilité d'absorption par neutron (par unité de temps). On 
considère donc un $\Sigma_a\varphi$ (où $\Sigma_a$ est constant car la région est homogène). Nous 
devons intégrer sur tout le volume d$\bar r$ et toutes les direction $\bar\Omega$ au sein du 
volume pour des neutrons qui ont été émis sur la surface c'est à dire entrante, d'où la présence 
du vecteur normal $\bar n$ de la surface extérieure et $\bar\Omega_s$ le courant.\\

Il reste à trouver l'intensité de la source tel que l'on ai un et un seul neutron par unité de 
temps (implicite) qui va donner une probabilité. Reprenons la probabilité de fuite
\begin{equation}
\begin{array}{ll}
\DS {P_o} = \frac{1}{{4\pi V}}\int\limits_S    \int\limits_{\bar n.\bar \Omega  > 0}    \int\limits_V   \int\limits_{{{\bar \Omega }_o}}&  \DS  \varphi ({\bar r_o}, - {\bar \Omega _o}|{\bar r_s}, - \bar \Omega )d{\bar r_o}d{\bar \Omega _o}\bar n.\bar \Omega d\bar \Omega dS\\ & \DS = \frac{S}{{4V}}.\frac{1}{{\pi S}}\int\limits_S    \int\limits_{\bar n.\bar \Omega  < 0}    \int\limits_V    \int\limits_{{{\bar \Omega }_o}}    \varphi ({\bar r_o},{\bar \Omega _o}|{\bar r_s},\bar \Omega )d{\bar r_o}d{\bar \Omega _o}\bar n.\bar \Omega d\bar \Omega dS
\end{array}
\end{equation}
Sachant que $d\bar\Omega_s = \in\theta d\theta d\varphi)$, il faut intégrer ça de $0$ à $\pi$ sur 
$d\varphi$ et puis intégrer de 0 à $\pi/2$. L'intégration sur $d\varphi$ donne 2$\pi$. Sachant que 
$d\cos\theta = \sin\theta d\theta$, l'intégration donne $2\pi x^2/2$ entre 0 et 1 si $x=\cos\theta$. 
Ceci donne $\pi$. Voici comment calculer l'intégrale ci-dessus. On en tire une expression 
entre les deux probabilités\\

\cadre{\begin{equation}
{\Gamma _o} = {\Sigma _a}\frac{{4V}}{S}{P_o}
\end{equation}}


\newpage
\section{Approximation de la diffusion}
\subsection{Équation de continuité}
L'idée est que comme l'anisotropie est assez faible sur certains type de réacteur, il doit être 
possible de se débarrasser de la dépendance de la direction angulaire. Pour se faire, on 
intègre l'équation de Boltzmann sur toutes les directions
\begin{equation}
\div(\bar J(\bar r,v)) + {\Sigma _t}(\bar r,v)\varphi (\bar r,v) - \int_o^\infty     {\Sigma _s}(\bar r,v' \to v)\varphi (\bar r,v')dv'  = \chi (v)\int_o^\infty     \nu {\Sigma _f}(\bar r,v')\varphi (\bar r,v')dv' + Q(\bar r,v)
\end{equation}
\textit{Justifications} :
\begin{itemize}
\item[$\bullet$] $\Sigma_t\varphi$ c'est juste le flux sur $d\bar\Omega$
\item[$\bullet$] On peut faire apparaître la divergence du courant $\bar{\Omega}\varphi$
\item[$\bullet$] Le terme en $\Sigma_s$ ne dépend que de la déflection angulaire. Comme on intègre 
sur toutes les $\bar\Omega$ pour un $\bar{\Omega}'$ donné ce qui donné le scattering différentiel 
en vitesse, encore une fois parce que toutes les vitesses ont été intégrées.
\end{itemize}\ \\

L'équation obtenue ne dépend plus que de $\bar r$ et $v$\dots mais aussi de $\bar\Omega$, ce n'est 
pas encore aussi compacté que ça : c'est plus simple, mais ce n'est pas bon pour déterminer le 
flux uniquement avec $\bar r$ et $v$. Le souci est le terme de courant qui contient implicitement 
du $\bar\Omega$
\begin{equation}
\bar J(\bar r,v) = \int\limits_{4\pi }   \bar \Omega \varphi (\bar r,v,\bar \Omega )d\bar \Omega 
\end{equation}
Nous allons chercher à établir une unique équation, et ce grâce à une équation constitutive.

\subsection{Équation de diffusion}
Une équation de continuité doit lier le gradient du flux intégré $\varphi(\bar r,v)$ au courant 
$\bar J(\bar r, v)$. Dans l'idée de supprimer les deux dépendances aux variables angulaires dans 
l'équation de transport, on va postuler une équivalence générale de la loi de Fick
\begin{equation}
\bar J(\bar r,v) =  - D(\bar r,v)\bar \nabla \varphi (\bar r,v)
\end{equation}
où $D(\vec r, v)$ est le coefficient de diffusion (unité de longueur) sur lequel nous n'avons 
aucune information, c'est juste une hypothèse. En utilisant cette loi constitutive (sans 
justification), on obtient
\begin{equation}
\begin{array}{ll}
\DS- \bar \nabla (D(\bar r,v)\bar \nabla \varphi (\bar r,v)) &\DS+ {\Sigma _t}(\bar r,v)\varphi (\bar r,v) - \int_o^\infty    {\Sigma _s}(\bar r,v' \to v)\varphi (\bar r,v')dv'\\
&\DS = \chi (v)\int_o^\infty    \nu {\Sigma _f}(\bar r,v')\varphi (\bar r,v')dv' + Q(\bar r,v)
\end{array}
\end{equation}
Il s'agit d'une équation du second ordre qui défini le flux uniquement en fonction de $\vec r$ et $v$.


\subsection{Conditions aux limites}
Nous avions introduit les conditions aux limites de vide sur l'équation de transport en annonçant 
que le flux en périphérie d'un réacteur convexe s'annule pour toutes les directions entrantes ; le 
flux net entrant dans le réacteur est nul. Le souci est que ces CL dépendent de $\bar\Omega$ que 
l'on vient de gommer. \\

Avant d'en venir à la CL, regardons la continuité en intégrant l'équation sur un petit volume 
autour d'une discontinuité (sans sources superficielles)
\begin{equation}
\int\limits_{{V_\varepsilon }}   div(\bar J(\bar r,v))dV = 0
\end{equation}
Sur un volume élémentaire le divergence est constante, sors de l'intégrale, on multiplie par $V_s$ 
qui tend vers zéro, ça fait bien zéro. On en tire que la continuité de la composante normale 
du courant ${J_n}({\bar r_s},v) \equiv \bar n.\bar J({\bar r_s},v)$ est continue. Dès lors, il faut 
avoir la dérivée normale du flux\footnote{Le courant doit être continu la dérivée normale du flux
peut être discontinue si l'on a pas le même coefficient de diffusion de part et d'autre}
\begin{equation}
{D_ + }\frac{{\partial {\varphi _ + }({{\bar r}_s},v)}}{{\partial n}} = {D_ - }\frac{{\partial {\varphi _ - }({{\bar r}_s},v)}}{{\partial n}}
\end{equation}
En montrant la continuité de la dérivée tangentielle du flux, on démontre la continuité du flux
\begin{equation}
\varphi ({\bar r_s} + \varepsilon \bar n,v) - \varphi ({\bar r_s} - \varepsilon \bar n,v) = \int_{ - \varepsilon }^{ + \varepsilon }   \frac{{\partial {\varphi _ - }({{\bar r}_s} + \xi \bar n,v)}}{{\partial n}}d\xi 0
\end{equation}
Le courant entrant dans le réacteur (le courant net, soit ce qui entre - ce qui sort) est nul
\begin{equation}
{J_ - } =  - \int\limits_{\bar n.\bar \Omega  < 0}   \bar n.\bar \Omega \varphi ({\bar r_s},v,\bar \Omega )d\bar \Omega  = 0
\end{equation}
Dans l'hypothèse de l'anisotropie faible, faisons un développement au premier ordre 
\begin{equation}
\varphi (\bar r,v,\bar \Omega ) = \frac{1}{{4\pi }}({\varphi _o}(\bar r,v) + \bar \Omega .{\bar \varphi _1}(\bar r,v))
\end{equation}
Le terme $1/4\pi$ est la pour prendre de l'avance, $\bar\Omega\bar\varphi_i$ est une contribution 
non-identifiée. Nous pouvons juste dire qu'à partir du développement du flux angulaire au premier 
ordre, nous devrions avoir une conservation du flux et du courant intégré. Tentons de justifier 
cette expression\footnote{Il faudrait vérifier avec des  notes manuscrites.}
\begin{equation}
\int_{4\pi} \bar\Omega \bar\varphi(\bar r, v,\bar{\Omega})\ d\bar{\Omega} = \int_{4\pi} \varphi_0\ 
d\bar{\Omega} + 0 = \varphi_0(\bar r, v)= \varphi'(\bar r, v)
\end{equation}
Le terme nul vient de l'intégration de $\cos$ sur une période. Dans le second cas, nous avons
\begin{equation}
\int_{4\pi} \bar\Omega \bar\varphi(\bar r, v, \Omega) = 0 + \frac{1}{4\pi}\int_{4\pi} 
\bar \Omega (\bar\Omega\bar{\varphi_1}(\bar{r},v)\ d\bar \Omega
\end{equation}
Le terme nul vient de l'intégration d'une fonction impaire. En considérant un flux orienté selon
$\vec{1_z}$
\begin{equation}
\int_{4\pi} \Omega_z(\Omega_x\varphi_{1x}+\Omega_x\varphi_{1y}+\Omega_x\varphi_{1z})\ d\bar{\Omega}
\end{equation}
Il va alors rester
\begin{equation}
\frac{1}{4\pi} \int_{4\pi} \Omega_z^2\varphi_{1z}\ d\bar{\Omega}
\end{equation}
La composante $\Omega_z^2$ fait intervenir un $\cos^2\theta$ qui va simplifier les $4\pi$ après 
intégration. De même
\begin{equation}
\frac{1}{2}\int_{-\pi/2}^{\pi/2} \cos^2\theta\sin\theta d\theta = \frac{1}{2}\int_{-1}^1\frac{x^3}{3}
\varphi_{1z}\ dx = \frac{\varphi_{1z}}{3} = J_z(\vec{r},v)
\end{equation}
En appliquant ce raisonnement pour les autres directions, on comprend le facteur trois devant 
la composante $\bar{\Omega}\bar{J}$, le flux devant être le même pour les autres directions. Ceci 
étant fait, il est possible d'exprimer les courants partiels 
\begin{equation}
\left\{\begin{array}{ll}
\DS {J_ + } =  + \int\limits_{\bar n.\bar \Omega  > 0}   \bar n.\bar \Omega \varphi (\bar r,v,\bar \Omega )d\bar \Omega  = \frac{1}{4}{\varphi _o}(\bar r,v) + \frac{1}{6}{\varphi _{1n}}(\bar r,v) &=\DS
 \frac{1}{4}\varphi (\bar r,v) - \frac{1}{2}D\bar n.\bar \nabla \varphi (\bar r,v)\vspace{2mm}\\
\DS{J_ - } =  - \int\limits_{\bar n.\bar \Omega  < 0}   \bar n.\bar \Omega \varphi (\bar r,v,\bar \Omega )d\bar \Omega  = \frac{1}{4}{\varphi _o}(\bar r,v) - \frac{1}{6}{\varphi _{1n}}(\bar r,v) 
&=\DS \frac{1}{4}\varphi (\bar r,v) + \frac{1}{2}D\bar n.\bar \nabla \varphi (\bar r,v)
\end{array}\right.
\end{equation}
où ${\varphi _{1n}}(\bar r,v) = \bar n.{\bar \varphi _1}(\bar r,v)$. Justifions avec, pour 
simplifier, $\bar n = \vec{1_z}$.
\begin{equation}
J_+ = \int_{\bar n.\bar{\Omega}>0} \Omega_z\frac{1}{4\pi}(\varphi(\bar r,v)+3\bar\Omega\bar{J}
(\bar r,v))\ d\bar{\Omega}
\end{equation}
Pour la direction sortante, on peut mettre des bornes\footnote{Je vois pas trop d'où sortent les $\cos\theta$}
\begin{equation}
J_+ = \int_0^{2\pi} \int_0^{\pi/2} d\varphi d\cos\theta \left(\cos\theta\frac{1}{4\pi} \varphi(\bar r,v) + \cos\theta\left(\frac{3}{4\pi}\Omega_z J_z(\bar r,v)\right)\right)
\end{equation}
Ce qui donne (les bornes sont prise pour ne considérer que la direction sortante)
\begin{equation}
\frac{1}{2}\int_0^1 \varphi(\bar r,v) \dfrac{x^2}{2}dx  + \frac{3}{2}J_z(\bar r,v)\left[
\frac{x^3}{3}\right]_0^1
\end{equation}
Avec Fick, on montre alors que le courant partiel entrant est donné par $1/4\dots$\footnote{Je ne 
vois pas du tout l'intérêt de la dernière ligne}.\\

Le courant partiel entrant s'annule à la frontière (${J_ - }({\bar r_s},v) = 0$) si 
\begin{equation}
 - \frac{{{\varphi _n}'({{\bar r}_s},v)}}{{\varphi ({{\bar r}_s},v)}} = \frac{1}{{2D({{\bar r}_s},v)}}
\end{equation}
En diffusion, le courant est donné par la loi de Fick. Mathématiquement, on exprime la nullité du flux
en $d_e = 2D(\bar r_s, v)$ qui est la \textbf{frontière extrapolée} (approximation linéaire du flux 
en dehors du réacteur et donc de la dérivée).


\subsection{Conditions de validités}
Nous utilisons pour l'instant un coefficient $D$ sans vraiment savoir de quoi il s'agit. La seul 
hypothèse implicite est que ce coefficient est la à cause de la diffusion d'un milieu matériel. Si 
le libre parcours moyen est petit par rapport à la dimension du milieu considéré, cette diffusion 
va être globalement de l'ordre de ce libre parcours moyen.  Par contre, proche des frontières, 
l'approximation sera moins bonne.



\subsection{Approximation $P_1$ à une vitesse de diffusion}
En considérant une anisotropie au premier ordre mono-cinétique
\begin{equation}
\varphi (\bar r,\bar \Omega ) = \frac{1}{{4\pi }}(\varphi (\bar r) + 3\bar \Omega .\bar J(\bar r))
\label{eq:Ch3.4}
\end{equation}
L'équation de transport à une vitesse s'écrit 
\begin{equation}
\bar \Omega .\bar \nabla \varphi (\bar r,\bar \Omega ) + {\Sigma _t}(\bar r)\varphi (\bar r,\bar \Omega ) - \frac{{c{\Sigma _t}(\bar r)}}{{2\pi }}\int\limits_{4\pi }   f(\bar \Omega '.\bar \Omega )\varphi (\bar r,\bar \Omega ')d\bar \Omega ' = Q(\bar r,\bar \Omega )
\label{eq:Ch3.5}
\end{equation}
où $f$ est la probabilité de déflection angulaire lors de la diffusion. En substituant 
\eqref{eq:Ch3.5} dans \eqref{eq:Ch3.4} et prenons-en le moment d'ordre 0\footnote{Pour 
rappel, le moment à l'ordre $r$ : $m_r(f) = \int x^r f(x)dx$.}\ \\

\cadre{
\begin{equation}
div{\kern 1pt} \bar J(\bar r) + (1 - c){\Sigma _t}(\bar r)\varphi (\bar r) = Q(\bar r)
\end{equation}}\ \\

L'intégration fait apparaître la divergence de $\bar J$, $\Sigma_t\varphi$ va donner le 
flux intégrer et pour le terme en courant, un $\bar{\Omega}$ intégré sur tous les 
$\bar\Omega$ s'annule. Pour le terme de fission de scattering ($\int_{4\pi}$), l'intégration 
sur tous les $\bar\Omega$ fait disparaître le $2\pi$ et le $\cos\theta$ donne 1. \\

Un peu plus de sport, le moment du premier ordre. Il faut multiplier l'équation de transport 
par $\bar\Omega$ et intégrer sur tous les $\bar\Omega$. Quelques préliminaires 
\begin{enumerate}
\item Résulte d'une intégration d'un sin ou cos, qui donne zéro.
\begin{equation}
\int\limits_{4\pi }   {\Omega _i}d\bar \Omega  = 0\quad ,\;i = x,y,z
\end{equation}
\item "Non vu, refaire calmement l'intégrale"
\begin{equation}
\int\limits_{4\pi }    {\Omega _i}{\Omega _j}d\bar \Omega  = \frac{{4\pi }}{3}{\delta _{ij}}\quad ,\;i,j = x,y,z
\end{equation}
\item\begin{equation}
\int\limits_{4\pi }    {\Omega _i}{\Omega _j}{\Omega _k}d\bar \Omega  = 0\quad ,\;i,j,k = x,y,z
\end{equation}
\end{enumerate}
En conséquence, il en vient que 
\begin{itemize}
\item[$\triangleright$]
\begin{equation}
\int\limits_{4\pi }   {\Omega _x}\bar \Omega .\bar \nabla \varphi (\bar r,\bar \Omega )d\bar \Omega\overset{P_1}{\longrightarrow} \frac{1}{3}\frac{{\partial \varphi (\bar r)}}{{\partial x}}
\end{equation}
\begin{proof}
\begin{equation}
\int_{4\pi} \bar{\Omega}(\bar{\Omega}\bar\nabla\varphi(\bar r,\bar\Omega))\ d\Omega
\end{equation}
En regardant uniquement pour $z$
\begin{equation}
\int_{4\pi} \Omega_z(\Omega_x\partial_x\varphi +\Omega_y\partial_y\varphi+\Omega_z\partial_z\varphi)\ d\bar\Omega =\int_{4^pi} \Omega_z^2 \partial_z\varphi d\bar{\Omega}
\end{equation}
En sortant $\partial_z\varphi$ qui ne dépend pas de $\bar\Omega$ on obtient $\dfrac{\partial_z\phi}{3}$.
\end{proof}

\item[$\triangleright$]
\begin{equation}
\int\limits_{4\pi }    {\Omega _x}{\Sigma _t}(\bar r)\varphi (\bar r,\bar \Omega )d\bar \Omega\overset{P_1}{\longrightarrow} {\Sigma _t}(\bar r){J_x}(\bar r)
\end{equation}
\item[$\triangleright$]
\begin{equation}
\int\limits_{4\pi }    \int\limits_{4\pi }    {\Omega _x}f(\bar \Omega .\bar \Omega ').\varphi (\bar r,\bar \Omega ')d\bar \Omega d\bar \Omega ' \overset{P_1}{\longrightarrow} ??
\end{equation}
\end{itemize}
Nous avions en effet vu que l'on pouvait utiliser les polynômes de Legendre pour exprimer
\begin{equation}
f(\mu ) = \sum\limits_l   \frac{{2l + 1}}{2}{f_l}{P_l}(\mu )\quad \quad ,\quad \mu  = \bar \Omega .\bar \Omega '
\end{equation}
On va utiliser le \textit{théorème d'addition des polynômes de Legendre} : 
\begin{equation}
{P_l}(\bar \Omega .\bar \Omega ') = {P_l}(\bar \Omega .\bar n).{P_l}(\bar \Omega '.\bar n) + \sum\limits_{\begin{array}{*{20}{c}}
{m =  - l}\\
{m \ne 0}
\end{array}}^{ + l}    {e^{im\phi }}...
\end{equation}
Le problème est le terme de sommation supplémentaire qui n'est pas très joli mais non gênant : il 
donnera un cos/sin qui, intégré sur la sphère unité, vaudra zéro. Pour des raisons d'orthogonalité, 
seul $m=1$ devra être considéré dans la développement en série de $f(\mu)$
\begin{equation}
\int\limits_{4\pi }   \int\limits_{4\pi }   {\Omega _x}{P_l}(\bar \Omega .\bar \Omega ').\varphi (\bar r,\bar \Omega ')d\bar \Omega d\bar \Omega ' = \int\limits_{4\pi }   2\pi \frac{2}{3}\Omega {'_x}{\delta _{1l}}\varphi (\bar r,\bar \Omega ')d\bar \Omega ' = \frac{{4\pi }}{3}{\delta _{1l}}{J_x}(\bar r)
\end{equation}
Le moment du premier ordre vaut donc\\

\cadre{
\begin{equation}
\frac{1}{3}\frac{{\partial \varphi (\bar r)}}{{\partial x}} + {\Sigma _t}(\bar r)(1 - c{f_1}(\bar r)){J_x}(\bar r) = \int\limits_{4\pi }    {\Omega _x}Q(\bar r,\bar \Omega )d\bar \Omega 
\end{equation}}\ \\

Soit (1-nombre de neutron)\dots = élément de source extérieure. Le résultat se généralise facilement
à trois dimensions
\begin{equation}
\frac{1}{3}\bar \nabla \varphi (\bar r) + ({\Sigma _t}(\bar r) - {\Sigma _{s1}}(\bar r))\bar J(\bar r) = {\bar Q_1}(\bar r)
\end{equation}
où $\DS {Q_{1i}}(\bar r) = \int\limits_{4\pi }    {\Omega _i}Q(\bar r,\bar \Omega )d\bar \Omega \quad ,\;i = x,y,z$ et $\DS {\Sigma _{s1}}(\bar r) = c(\bar r){\Sigma _t}(\bar r){f_1}(\bar r) = c(\bar r){\Sigma _t}(\bar r) < {\mu _o} > $.\\

En considérant un matériau homogène et des sources isotopiques, on obtient une loi constitutive 
qui donne la loi de diffusion
\begin{equation}
\bar J(\bar r) =  - \frac{1}{{3({\Sigma _t} - {\Sigma _{s1}})}}\bar \nabla \varphi (\bar r)
\end{equation}
Ce qui n'est rien d'autre que la loi de Fick où $D = \frac{1}{{3({\Sigma _t} - {\Sigma _{s1}})}}$.\\

En considérant le moment du premier ordre du transport mono-énergétique développé au premier ordre, 
nous avons établi une première manière, dans ce contacte, de pouvoir relier le coefficient de 
diffusion à un $\Sigma_t$, une caractéristique neutronique ! En définissant la section efficace 
de transport ${\Sigma _{tr}} = {\Sigma _t} -  < {\mu _o} > {\Sigma _s}$, on peut approximer le 
coefficient de diffusion (\textbf{sans} fissions)
\begin{equation}
D = \frac{1}{{3{\Sigma _{tr}}}}
\end{equation}


\subsection{Solution de l'équation de diffusion monocinétique (sans fission)}
\subsubsection{Milieu infini}
Soit la diffusion à vitesse constante dans un milieu infini homogène ($D$ constant) avec un point
source en $O$
\begin{equation}
 - \bar \nabla (D(\bar r)\bar \nabla \varphi (\bar r)) + {\Sigma _t}(\bar r)\varphi (\bar r) - {\Sigma _s}(\bar r)\varphi (\bar r) = Q(\bar r)
\end{equation}
La source étant ponctuelle et les fissions nulles
\begin{equation}
 - D\Delta \varphi (\bar r) + {\Sigma _a}\varphi (\bar r) = Q\delta (\bar r)
\end{equation}
Posons ${\Sigma _a}/D = {\kappa ^2}$ de sorte à ré-écrire l'équation
\begin{equation}
 - \Delta \varphi (\bar r) + {\kappa ^2}\varphi (\bar r) = \frac{Q}{D}\delta (\bar r)
\end{equation}
A l'aide de la transformée de Fourier\footnote{Dans quel but ?}
\begin{equation}
\hat \varphi (\bar k) = {\left( {\frac{1}{{2\pi }}} \right)^{3/2}}\int   {e^{ - i\bar k.\bar r}}\varphi (\bar r)d\bar r\qquad\Rightarrow\qquad \hat \varphi (\bar k) = {\left( {\frac{1}{{2\pi }}} \right)^{3/2}}\frac{Q}{{D({\kappa ^2} + {k^2})}}
\end{equation}
On retrouve\footnote{Comment?} la fonction de Green
\begin{equation}
{\varphi _G}(\bar r) = \frac{{{e^{ - \kappa r}}}}{{4\pi Dr}}
\end{equation}
Pour une source générale
\begin{equation}
\varphi (\bar r) = \int\limits_{{R^3}}    \frac{{{e^{ - \kappa |\bar r - {{\bar r}_s}|}}}}{{4\pi D|\bar r - {{\bar r}_s}|}}Q({\bar r_s})d{\bar r_s}
\end{equation}
Nous obtenons une exponentielle décroissante mais nous avions obtenu en étudiant l'épaisseur optique
un facteur $1/r^2$ ce qui n'est plus le cas ici : l'approximation de diffusion change quelque peu 
la nature de la diffusion elle-même. Le slide 17 reprend quelques exemples de calcul, voir TP.

\subsubsection{Milieu fini}
Nous allons utiliser la méthodes des sources virtuelles : on va remplacer le problème possédant une 
source réelle et une limite physique par un problème qui a des sources réelles, des sources 
virtuelles à la périphérie le tout dans un milieu infini. Sur les bords, les sources virtuelles 
étant négatives cela crée de l'absorbance et donc une équivalence des situations. Ceci permet de 
satisfaire les CL.  Prenons l'exemple d'une source plane centrée (frontière extrapolée d'épaisseur 
$2a$)
\begin{equation}
Q(\bar r) = {Q_o}\delta (x)
\end{equation}
La CL de la frontière extrapolée s'exprime $\varphi ( \pm a) = 0$ et celle-ci est satisfaite par 
l'ajout de source virtuelles
\begin{equation}
{Q_v}(\bar r) = A\delta (x - a) + A\delta (x + a)
\end{equation}
Le flux total est donc induit par trois sources : la "vraie" et nos virtulles
\begin{equation}
\varphi (x) = \frac{1}{{2\kappa D}}({Q_o}{e^{ - \kappa |x|}} + A{e^{ - \kappa (x + a)}} + A{e^{ - \kappa (a - x)}})\quad ,\quad x \in [ - a, + a]
\end{equation}
En appliquant les CL on trouve
\begin{equation}
\varphi (x) = {Q_o}\frac{{\sinh \kappa (a - |x|)}}{{2\kappa D\cosh \kappa a}}\quad  < {\varphi _\infty }(x)\;,\quad x \in [ - a, + a]
\end{equation}
Le slide 19 donne l'exemple d'une source uniforme.




\subsubsection{Longueur de diffusion}
Pour une source plane, nous avions une idée du temps de relaxation qui est la longueur sur laquelle 
la réaction va se produire. Nous avons vu que celui-ci devait se rapprocher du libre parcours moyen. 
En définissant la \textbf{longueur de diffusion}
\begin{equation}
L = \dfrac{1}{\kappa}
\end{equation}
où $\DS {L^2} = \frac{D}{{{\Sigma _a}}} =  = \frac{1}{{3{\Sigma _{tr}}{\Sigma _a}}} = \frac{{ < {\lambda _{tr}} >  < {\lambda _a} > }}{3}$ nous voyons que $L$ fait bien intervenir $D$ mais aussi
des caractéristiques neutroniques.  Pour une source plane, on trouverait
\begin{equation}
\varphi (x) = \frac{L}{{2D}}{e^{ - \frac{{|x|}}{L}}}
\end{equation}
où $L$ est la longueur de relaxation. \\

Regardons la distance quadratique moyenne à la source ponctuelle. Pour se faire, considérons cette 
distance $r^2$, pondérons-la par la source et intégrons sur une calotte sphérique pour enfin 
normaliser. En faisant apparaître la fonction gamma, on trouve
\begin{equation}
\DS < {r^2} > \, = \dfrac{{\int_o^\infty     {r^2}\varphi (r)4\pi {r^2}dr}}{{\int_o^\infty     \varphi (r)4\pi {r^2}dr}} = \dfrac{{\int_o^\infty     {r^3}{e^{ - \kappa r}}dr}}{{\int_o^\infty    r{e^{ - \kappa r}}dr}} = 6{L^2}
\end{equation}
Il s'agit de la distance sur laquelle le flanc s'étale en présence d'une source ponctuelle.


\section{Approximation multi-groupe}
\subsection{Groupe d'énergie}
Faire l'approximation mono-cinétique n'est pas réaliste ($E \in [10^{-2}, 10^6]$ eV). Pour tout de 
même garder notre approximation précédente, nous allons discrétiser l'énergie en $G$ groupes
\begin{equation}
E_G < \dots < E_g < \dots < E_0
\end{equation}
avec $E_0$ qui correspond aux neutrons rapide et $E_G$ aux thermiques. On va ainsi travailler avec un 
ensemble moyenné sur toute ces intervalles d'énergie mais ce qu'il faut retenir c'est que la 
dépendance en énergie est supprimée mais nous avons autant d'équations qu'il y a de groupes
\begin{equation}
{\varphi _g}(\bar r) = \int_{{E_g}}^{{E_{g - 1}}}    \varphi (\bar r,E)dE = \int\limits_g    \varphi (\bar r,E)dE\quad ,\;g = 1,\dots, G
\end{equation}
On essaye de conserver le taux de réaction. Lorsqu'on intègre $\Sigma_t\varphi(\bar r,E)$ sur 
un groupe on peut dire que c'est $\Sigma_{tg}\varphi_g$ où $\Sigma_{tg}$ défini la\textit{ section  efficace totale} dans le groupe $g$. Ainsi, le taux de réaction dans $g$ va définir la section totale $\Sigma_{tg}$
\begin{equation}
{\Sigma _{tg}}(\bar r) = \frac{1}{{{\varphi _g}(\bar r)}}\int\limits_g    {\Sigma _t}(\bar r,E)\varphi (\bar r,E)dE
\end{equation}
Il s'agit très clairement de l'expression d'une moyenne.\\

De façon similaire, nous allons avoir un \textit{coefficient de diffusion }moyen pour chaque groupe $g$ qui dépendra de chacune des directions $x$ que l'on peut avoir
\begin{equation}
{D_{gx}}(\bar r) = \frac{1}{{{\textstyle{{\partial {\varphi _g}(\bar r)} \over {\partial x}}}}}\int\limits_g   D(\bar r,E)\frac{{\partial \varphi (\bar r,E)}}{{\partial x}}dE
\end{equation}
Il est possible de perdre l'isotropie ! Dans un cas isotropique : $\DS{\bar J_g}(\bar r) =  - {D_g}(\bar r)\bar \nabla {\varphi _g}(\bar r)$.\\

Il est possible de définir la \textit{section efficace de transfert} par scattering entre deux groupes 
(simple généralisation du scattering où les énergies de départ et d'arrivées sont dans un 
continuum) soit le passage d'un groupe $g'$ à un groupe $g$ :
\begin{equation}
{\Sigma _{sg'g}}(\bar r) = \frac{1}{{{\varphi _{g'}}(\bar r)}}\int\limits_g    \int\limits_{g'}    {\Sigma _s}(\bar r,E' \to E)\varphi (\bar r,E')dE'dE
\end{equation}
La \textit{fission} dans le groupe $g$ s'exprime
\begin{equation}
\nu {\Sigma _{fg}}(\bar r) = \frac{1}{{{\varphi _g}(\bar r)}}\int\limits_g   \nu {\Sigma _f}(\bar r,E)\varphi (\bar r,E)dE
\end{equation}
En généralisant le spectre mawellien à un groupe ${\chi _g} = \int\limits_g    \chi (E)dE$ ainsi 
que la \textit{source externe} ${Q_g}(\bar r) = \int\limits_g    Q(\bar r,E)dE$ on obtient en 
mettant tout ensemble les \textbf{équations de diffusion multi-groupes} :\\

\cadre{\begin{equation}
\begin{array}{ll}
\DS - \bar \nabla ({D_g}(\bar r)\bar \nabla {\varphi _g}(\bar r)) &\DS+ {\Sigma _{tg}}(\bar r){\varphi _g}(\bar r) - \sum\limits_{g' = 1}^G    {\Sigma _{sg'g}}(\bar r){\varphi _{g'}}(\bar r)\\
&\DS= {\chi _g}\sum\limits_{g' = 1}^G    \nu {\Sigma _{fg'}}(\bar r){\varphi _{g'}}(\bar r) + {Q_g}(\bar r)\quad ,\,g = 1,\dots,G
\end{array}
\end{equation}}\ \\

Nous avons ici un système de $G$ équations couplées, en grande partie à cause du scattering, sur les
$G$ groupes mais nous n'avons pas fait d'approximations supplémentaires. De cette sommation, on 
peut extraire le cas où $g=g'$ et interpréter ce terme comme une "extraction" de $g$ par tout type 
d'interaction. En l'isolant, on peut définir une \textit{section de removal} qui est la probabilité par unité de longueur qu'un neutron soit supprimé du groupe $g$
\begin{equation}
{\Sigma _{rg}}(\bar r) = {\Sigma _{tg}}(\bar r) - {\Sigma _{sgg}}(\bar r) = {\Sigma _{ag}}(\bar r) + \sum\limits_{g' \ne g}^{}    {\Sigma _{sgg'}}(\bar r)
\end{equation}
Encore une fois, aucune hypothèse supplémentaire n'a été faite.







\newpage
\section{Méthode de probabilité de première collision}
Le slide 24 (re)donne la (lourde) forme intégrale de l'équation de transport ainsi que le cas 
isotropique avec l'énergie en tant que variable. On va définir la \textit{distance optique} dans le
groupe $g$ comme le scattering au sein même de ce groupe
\begin{equation}
{\tau _{vg}}(\bar r,{\bar r_o}) \equiv \int_o^s    {\Sigma _{tg}}({\bar r_o} + s'\bar \Omega )ds'
\end{equation}
On peut alors établir l'équation suivante, qui possède une forme analogue à \eqref{eq:Ch3.cp}, 
établie pour le transport monoénergétique. Il s'agit des \textit{équations de transport 
multi-groupes} dans le cas isotropique
\begin{equation}
{\varphi _g}(\bar r) = \int_{{R^3}}^{} {  \frac{{{e^{ - {\tau _{vg}}(\bar r,{{\bar r}_o})}}}}{{4\pi {{\left| {\bar r - {{\bar r}_o}} \right|}^2}}}} ({\Sigma _{sgg}}({\bar r_o}){\varphi _g}({\bar r_o}) + {S_g}({\bar r_o}))d{\bar r_o}\quad ,\;g = 1,\dots,G
\end{equation}
Il s'agit d'un noyau de transport que multiplie un source extérieure $Q$ et un autre terme de 
scattering. Pour le cas multigroupe, il faut remplacer (par rapport à \eqref{eq:Ch3.cp}) $\Sigma_t$
par $\Sigma_{sgg}$ (scattering au sein d'un même groupe d'énergie) ainsi qu'une source extérieure qui
comprend toutes les contributions liées à $g$
\begin{equation}
{S_g}(\bar r) = \nu {\chi _g}\sum\limits_{g'}^{}    {\Sigma _{fg'}}(\bar r){\varphi _{g'}}(\bar r) + {Q_g}(\bar r) + \sum\limits_{g' = 1}^{g - 1}   {\Sigma _{sg'g}}(\bar r){\varphi _{g'}}(\bar r)
\end{equation}
Ce terme prend donc en compte les contributions extérieures : fission qui se passent un peu partout 
(le dernier terme est le scattering des autres groupes pour le ralentissement des neutrons). Les 
équations ont la même forme que le cas monoénergétique. Par similitude, les théorème de réciprocité 
reste valable, son corollaire également, \dots


\section{Implémentation de la méthode de probabilité de première collision}
La méthode de probabilité de première collision revient à partir d'une équation de Boltzmann 
mono-énergétique pour chaque groupe
\begin{equation}
\varphi (\bar r) = \int_{{R^3}}^{} {  K(\bar r,{{\bar r}_o})} ({\Sigma _s}({\bar r_o})\varphi ({\bar r_o}) + S({\bar r_o}))d{\bar r_o}  \equiv \int_{{R^3}}^{} {  K(\bar r,{{\bar r}_o})} {Q_t}({\bar r_o})d{\bar r_o}
\end{equation}
où $S$ contient les différentes sources et 
\begin{equation}
K(\bar r,{\bar r_o}) = \frac{{{e^{ - {\tau _v}(\bar r,{{\bar r}_o})}}}}{{4\pi {{\left| {\bar r - {{\bar r}_o}} \right|}^2}}}
\end{equation}
Nous allons partitionner le réacteur en petits volumes $V_i$ homogènes où le flux est constant (
hypothèse du flux plat). Multiplions l'équation de Boltzmann par $\Sigma_t$ et intégrons sur ces 
volumes
\begin{equation}
\int\limits_{{V_i}}    {\Sigma _t}(\bar r)\varphi (\bar r)d\bar r = \sum\limits_j    \int\limits_{{V_j}}    {Q_t}({\bar r_o})d{\bar r_o}\int\limits_{{V_i}}   {\Sigma _t}(\bar r)K(\bar r,{\bar r_o})d\bar r
\end{equation}
Étant donné l'homogénéité des volumes et en notant $\varphi_i$ le flux moyen sur un volume $V_i$
\begin{equation}
\DS{V_i}{\Sigma _{ti}}{\varphi _i} = \sum\limits_j    \bar P_{j \to i}^{1t}{V_j}{Q_{tj}}\quad\text{ avec }\quad \left\{\begin{array}{ll}
\DS\varphi _i^{} &\DS= \frac{1}{{{V_i}}}\int\limits_{{V_i}}    \varphi (\bar r)d\bar r\;,
Q_{ti}^{} = \frac{1}{{{V_i}}}\int\limits_{{V_i}}    {Q_t}(\bar r)d\bar r\vspace{2mm}\\
\DS\bar P_{j \to i}^{1t} &\DS= \frac{{\int\limits_{{V_j}}    d{{\bar r}_o}{Q_t}({{\bar r}_o})\int\limits_{{V_i}}    {\Sigma _t}(\bar r)K(\bar r,{{\bar r}_o})d\bar r}}{{\int\limits_{{V_j}}    {Q_t}({{\bar r}_o})d{{\bar r}_o}}}
\end{array}\right.
\label{eq:Ch3.6}
\end{equation}
Nous avons quelque chose qui ressemble fortement ($P_j$) à ce que nous avions introduit avec le 
théorème de réciprocité. On retrouve le noyau de transport $K$ donnant le flux à la première collision 
que l'on multiplie par $\Sigma_t$ pour avoir le taux de réaction. Celui-ci concerne des neutrons qui
proviennent de $\bar r_0$ dans le volume $V_j$ et qui vont subir une première collision dans le 
volume $V_i$ le tout pondéré par l'intensité de la source dans le volume $V_j$ et normalisé. Ceci 
est quasiment ce que nous avions obtenu mais avec une barre pour indiquer la valeur moyenne de la 
source.\\

Si la source est uniforme, il n'y a plus de dépendance en $\bar r_0$ et on se retrouve avec du 
1/(volume de départ). L'intérêt est que le système d'équations algébrique \eqref{eq:Ch3.6} devient 
assez simple lorsqu'on l'explicite dans un petit volume $j$, grâce à l'hypothèse d'uniformité.
\begin{equation}
{V_i}{\Sigma _{ti}}{\varphi _i} = \sum\limits_j    P_{j \to i}^{1t}{V_j}({\Sigma _{sj}}{\varphi _j} + {S_j})
\end{equation}
Il ne reste qu'à définir les probabilités $P$ de première collision entre deux volumes donnés mais 
ceci dépendant de la géométrie, il n'est pas possible de l'écrire de façon générale.\\

Lors de la résolution, si le volume est infini on peut en tirer quelques simplifications grâce à la
conservation des probabilités : si infini, la probabilité qu'un neutron dans un volume ai une 
interaction avec n'importe quel autre volume est 1
\begin{equation}
\sum\limits_j    P_{i \to j}^{1t} = 1
\end{equation}
Dans le vide, il faut compléter par la probabilité de fuite
\begin{equation}
\sum\limits_j    P_{i \to j}^{1t} + {P_{io}} = 1
\end{equation}
où $P_{io}$ est la probabilité de fuite dans un réacteur sans collision pour un neutron apparaissant 
dans $V_i$. Pour un réacteur fini
\begin{equation}
\sum\limits_j   P_{i \to j}^{1t} + {P_{iS}} = 1
\end{equation}
où $P_{iS}$ est la probabilité de fuite à travers une surface $S$ du réacteur, sans collision, pour 
un neutron apparaissant dans $V_i$. Le slide 30 applique le théorème de réciprocité et le slide 31 est cadeau ! 