\chapter{Interaction matière-lumière}
\section{Équations de Maxwell}
Les équations de \textsc{Maxwell} du champ électrique et du champ d'induction magnétique sont 
désormais bien connue
\begin{equation}

\end{equation}
La jauge de \textsc{Coulomb} $\vec\nabla.\vec{A}=0$ permet d'imposer des ondes transverses. On obtient
alors l'équation d'onde
\begin{equation}

\end{equation}
Nous avons en effet $\phi=0$ dans le vide. La solution de cette équation est une onde plane
monochromatique
\begin{equation}

\end{equation}
où $A_0$ décrit l'intensité et la polarisation de la radiation, $\vec{k}$ est le vecteur de 
propagation ($\omega=kc$) et où $\delta_\omega$ est une phase réelle. Comme annoncé, la jauge de
\textsc{Coulomb} impose
\begin{equation}

\end{equation}
Les ondes sont donc transverses : $\vec{k}\perp\vec{A_0}(\omega)$. Avec ce choix de potentiel vecteur,
nous pouvons ré-écrire le champ électrique, l'induction magnétique 
\begin{equation}

\end{equation}
et le potentiel vecteur
\begin{equation}

\end{equation}
où $\vec{\hat{\epsilon}}$ est le vecteur de polarisation. Il nous informe la direction dans laquelle
$\vec{A_0}$ pointe. Comme $\vec{E}$ contient également ce vecteur, il est forcément colinéaire à 
$\vec{A_0}$. L'onde est forcément transverse ($\vec{k}\perp\vec{A}$) et la direction de polarisation
de $\vec{E}$ est imposée par $\vec{A}$. Notons qu'il est possible de déterminer un état arbitraire
de polarisation en effectuant la combinaison de deux ondes planes indépendantes.\\

A l'aide du vecteur de \textsc{Poynting} ($\propto |E|^2$), il est possible de déterminer l'intensité
du champ électrique (ou du potentiel vecteur)
\begin{equation}

\end{equation}
La dernière égalité est le produit de la densité photonique par la vitesse de la lumière. La densité
photonique s'exprime
\begin{equation}

\end{equation}
Cette densité est l'énergie $\hbar\omega$ que porte chaque photon, multiplié par $N(\omega)$ le nombre
de photon, par la vitesse de la lumière $c$ et divisé par le volume. Il s'agit bien d'une énergie par
unité de surface et de temps.
\begin{equation}

\end{equation}
Notons les deux relations intégrales suivantes
\begin{equation}

\end{equation}

\section{Équation de Schrödinger dépendante du temps}
Les solutions des équations de \textsc{Maxwell} pouvant bien évidemment dépendre du temps, il faut se
soucier de l'évolution dans le temps (mais aussi toujours de l'espace) du potentiel vecteur en tenant
compte du fait que ce n'est pas intégralement monochromatique. On défini alors la forme générale d'un
pulse de radiation
\begin{equation}

\end{equation}
L'



\iffalse
intensité = densité photonique * vit lumiere

densité c'est une énergie hbar w que porte chaque photon * N(w) le nombre de photon, * c/ volume : on a une énergie par unité de surface et de temps
\fi