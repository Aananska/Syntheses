%Auteurs : Nicolas Englebert
\documentclass[british,french,11pt, a4paper, openany]{book}

% Règles de bonne pratiques :
% https://fr.wikibooks.org/wiki/LaTeX/Gestion_des_gros_documents
\usepackage{../../Builder/preambule}
% %%%%%%%%%%%%%%%%
%%% Packages %%%
%%%%%%%%%%%%%%%%

%%% Compatibilité %%%
\begingroup\expandafter\expandafter\expandafter\endgroup
\expandafter\ifx\csname IncludeInRelease\endcsname\relax
\usepackage{fixltx2e}
\fi 					% Si version LaTeX < 2015, inclut un fix.

%%% Général %%%
\usepackage[utf8]{inputenc}
\usepackage{babel}
\usepackage{lmodern}
\usepackage[T1]{fontenc}
\addto\extrasfrench{\sisetup{locale = FR,detect-all}} % Switch siunitx en fonction de la langue babel :)
\addto\extrasbritish{\sisetup{locale = UK,detect-all}}
\usepackage{courier}
\usepackage{graphicx}
%\usepackage{cancel}

%%% Tableau %%%
%\usepackage{tabularx} %Permet d'auto dimensionner les tableaux



%%% Bibliographie %%%
%\usepackage[style=alphabetic,backend=biber]{biblatex}
\usepackage[autostyle]{csquotes}
%\DeclareNameAlias{sortname}{last-first}
%\DeclareFieldFormat{url}{\space\url{#1}}
%\DeclareNameAlias{labelname}{last-first}
%\addbibresource{sample.bib}


%%% Graphiques %%%
%\usepackage{tikz}
%\usepackage{pgfplots}
%\usepackage{circuitikz}

%%% Mise en page %%%
\usepackage{mathtools}
\usepackage{amssymb}
\usepackage{bbm}
\usepackage{amsthm}
%\usepackage[tt]{titlepic}% Centre le titre
%\usepackage{fancyhdr}   % Permet de modifier l'entête & footer
\usepackage{caption}     % Permet d'ajouter des légendes en images sans les mettre en float + dans la marge + ref vers le haut de l'envirronement
\usepackage{wrapfig}
\usepackage{fullpage}
%\usepackage{multicol}   % pour les liste sur plusieurs colonnes
%\usepackage{subfigure}  % alligne deux images cote a cote
\usepackage{float}      %permet de mettre du texte entre les figures grace a [H]. Génial! 
\usepackage{eso-pic}    % Fond d'écran page de garde
\usepackage{adjustbox}  % Empêche les box de sortir de la page


%%% Math %%%
%\usepackage{delarray} % Belles matrices
\usepackage{siunitx}


%%% Codes %%%
%\usepackage{listings}
%\usepackage[final]{pdfpages} %% Inclusion fichier pdf

%% Reference
\usepackage{hyperref}
%\renewcommand*{\figureautorefname}{fig.}
%\def\appendixautorefname{annexe}
%\def\tableautorefname{tab.}
%\renewcommand*{\chapterautorefname}{ch.}
%\newcommand{\subfigureautorefname}{\figureautorefname}



%%%%%%%%%%%%%%%%%
%%% Commandes %%%
%%%%%%%%%%%%%%%%%

%%% Physique %%%
\newcommand{\cst}{\text{cst}}
\newcommand{\D}{\partial}
\newcommand{\E}{\vec E}
\newcommand{\B}{\vec B}
\newcommand{\F}{\vec F}
\newcommand{\modu}[1]{|$#1$|}

%%% Math %%%
\newcommand{\oiint}{\int\!\!\!\!\!\!\! \:\!\subset\!\!\supset\!\!\!\!\!\!\!\int}
\newcommand{\rot}{\operatorname{\vec{rot}}}
\newcommand{\divv}{\operatorname{div}}
\newcommand{\phas}[1]{\underline{#1}}
\newcommand{\RE}{\text{Re}}
\newcommand{\ft}{\overset{\mathcal{F}}{\longleftrightarrow}}
\newcommand{\lt}{\overset{\mathcal{L}}{\longleftrightarrow}}
\newcommand{\DS}{\displaystyle}
\newcommand{\Tr}{\operatorname{Tr}}



%% Box
\shorthandon{:}
\newcommand{\theor}[1]{\adjustbox{minipage=\linewidth-2\fboxsep-2\fboxrule,fbox}{\textsc{\iflanguage{british}{Theorem}{Théorème}: }#1}}
\newcommand{\defi}[1]{\adjustbox{minipage=\linewidth-2\fboxsep-2\fboxrule,fbox}{\textsc{\iflanguage{british}{Definition}{Définition}: }#1}}
\newcommand{\lemme}[1]{\adjustbox{minipage=\linewidth-2\fboxsep-2\fboxrule,fbox}{\textsc{\iflanguage{british}{Lemma}{Lemme}: }#1}}
\newcommand{\prop}[1]{\adjustbox{minipage=\linewidth-2\fboxsep-2\fboxrule,fbox}{\textsc{\iflanguage{british}{Property}{Propriété}}\\ #1}}
\newcommand{\proposition}[1]{\adjustbox{minipage=\linewidth-2\fboxsep-2\fboxrule,fbox}{\textsc{Proposition}\\#1}}
\newcommand{\cadre}[1]{\adjustbox{minipage=\linewidth-2\fboxsep-2\fboxrule,fbox}{#1}}
\newcommand{\retenir}[1]{\adjustbox{minipage=\linewidth-2\fboxsep-2\fboxrule,fbox}{\textbf{\textit{\textsc{\iflanguage{british}{To remember}{À retenir}}: }}#1}}

\newcommand{\corollaire}[1]{\bigbreak\begin{tabular}{||c}
	\begin{minipage}{\textwidth}
		\textsc{\iflanguage{british}{Corollary}{Corollaire}: } \textit{#1}
	\end{minipage}
	\end{tabular}}
\newcommand{\exemple}[1]{\bigbreak\begin{tabular}{|c}
	\begin{minipage}{\textwidth}
		\textsc{\iflanguage{british}{Example}{Exemple}: } #1
	\end{minipage}%
	\end{tabular}}%
\shorthandoff{:}
    

%\pagestyle{headings} % Titre du ch et numéro page dans l'entete
%\renewcommand{\proofname}{Démonstration}
%\addto\captionsfrench{\def\tablename{Tableau}}


%%% Background %%%
\newcommand\BackgroundPic{%
	\put(0,0){%
		\parbox[b][\paperheight]{\paperwidth}{%
			\vfill
			\centering
			\includegraphics[width=\paperwidth,height=\paperheight,%
			keepaspectratio]{../../Builder/ulb.jpg}%
			\vfill
}}}

%%% Annexes Cedu %%%
%\usepackage{calrsfs}
%\DeclareMathAlphabet{\pazocal}{OMS}{zplm}{m}{n}
\usepackage{fourier-orns}

\setlength{\parindent}{0pt} 

%%% Attributs %%%
\newcommand*{\NomduCours}[2]{\def\cours{#1}\def\memo{#2}}
\newcommand*{\annee}[2]{\def\adebut{#1}\def\afin{#2}}

\newcounter{auteurcnt}
\newcommand\addauteur[2]{%
	\stepcounter{auteurcnt}%
	\csdef{auteur\theauteurcnt}{\mbox{#1~\textsc{#2}}}}
\newcommand\getauteur[1]{%
	\csuse{auteur#1}}

\newcounter{illustrateurcnt}
\newcommand\addillustrateur[2]{%
	\stepcounter{illustrateurcnt}%
	\csdef{illustrateur\theillustrateurcnt}{\mbox{#1~\textsc{#2}}}}
\newcommand\getillustrateur[1]{%
	\csuse{illustrateur#1}}

\newcounter{rappeltheocnt}
\newcommand\addrappeltheo[2]{%
	\stepcounter{rappeltheocnt}%
	\csdef{rappeltheo\therappeltheocnt}{\mbox{#1~\textsc{#2}}}}
\newcommand\getrappeltheo[1]{%
	\csuse{rappeltheo#1}}

\newcounter{professeurcnt}
\newcommand\addprofesseur[2]{%
	\stepcounter{professeurcnt}%
	\csdef{professeur\theprofesseurcnt}{\mbox{#1~\textsc{#2}}}}
\newcommand\getprofesseur[1]{%
	\csuse{professeur#1}}

\newcounter{iter}

% Attributs
\NomduCours{Algèbre linéaire}{MATH-H-101}
\addauteur{Nicolas}{Englebert}
\addprofesseur{Dominique}{Buset}
\annee{2013}{2014}

% Document
\begin{document}
\def\equationautorefname~#1\null{%
	(#1)\null
}
\newcommand{\pscal}[2]{\left\langle {#1} , {#2} \right\rangle}

%%%%%%%%%%%%%%%%%
% Préliminaires %
%%%%%%%%%%%%%%%%%
\frontmatter
\AddToShipoutPicture*{\BackgroundPic}

\begin{titlepage}
	\begin{center}	
			
		\newcommand{\HRule}{\rule{\linewidth}{0.5mm}}   			            %Titre en gros
		\includegraphics[width=0.55\textwidth]{../../Builder/titlepage/logo.pdf}~\\[1cm]				%Logo
			
			\textsc{\LARGE Université Libre de Bruxelles}\\[1.5cm]
			\textsc{\Large \iflanguage{british}{Summary}{Synthèse}}\\[0.5cm]
			
			\HRule \\[0.4cm]
			{ \huge \bfseries \cours \ \\\memo \\[0.4cm] }
			
			
			\HRule \\[1.5cm]
			\begin{minipage}[t]{0.6\textwidth}
				\begin{flushleft}%\large
					\emph{\iflanguage{british}{Author}{Auteur}\ifnum\theauteurcnt>1 s\fi:}\\
					\whileboolexpr
					{ test {\ifnumcomp{\value{iter}}{<}{\theauteurcnt}} }%
					{\stepcounter{iter}\getauteur{\theiter}\\}
					\setcounter{iter}{0}%
					\ifnum\theillustrateurcnt>0%
					\ \\
					\emph{Illustrations:}\\
					\whileboolexpr
					{ test {\ifnumcomp{\value{iter}}{<}{\theillustrateurcnt}} }%
					{\stepcounter{iter}\getillustrateur{\theiter}\\}%
					\setcounter{iter}{0}%
					\fi%
					\ifnum\therappeltheocnt>0%
					\ \\
					\emph{\iflanguage{british}{Reminders}{Rappels théoriques}:}\\
					\whileboolexpr
					{ test {\ifnumcomp{\value{iter}}{<}{\therappeltheocnt}} }%
					{\stepcounter{iter}\getrappeltheo{\theiter}\\}%
					\setcounter{iter}{0}%
					\fi%
				\end{flushleft}
			\end{minipage}%
			\begin{minipage}[t]{0.25\textwidth}
				%\begin{flushright}
				%\large
				\emph{\iflanguage{british}{Professor}{Professeur}\ifnum\theprofesseurcnt>1 s\fi:}
				\whileboolexpr
				{ test {\ifnumcomp{\value{iter}}{<}{\theprofesseurcnt}} }%
				{\\ \stepcounter{iter}\getprofesseur{\theiter}}%
				\setcounter{iter}{0}%
				%\end{flushright}
			\end{minipage}
			
			\vfill
			
			% Bottom of the page
			{\large \iflanguage{british}{Year}{Année} \adebut~-~\afin}
			
		\end{center}
	\end{titlepage}

\ \\[2cm]
{\Huge \bfseries Appel à contribution}\\[5mm]
\subsection*{Synthèse Open Source}
\begin{wrapfigure}[5]{l}{4.5cm}
	\includegraphics[scale=0.5]{../../Builder/git.png}
\end{wrapfigure}
Ce document est grandement inspiré de l’excellent cours donné 
par \ifnum\theprofesseurcnt=1 \getprofesseur{1} \else\whileboolexpr
{ test {\ifnumcomp{\value{iter}}{<}{\theprofesseurcnt-2}} }%
{\stepcounter{iter}\getprofesseur{\theiter}, }%
\stepcounter{iter}\getprofesseur{\theiter} et \stepcounter{iter}\getprofesseur{\theiter} \fi%
 à l’EPB (École Polytechnique de Bruxelles), faculté de l’ULB (Université 
Libre de Bruxelles). Il est écrit par les auteurs susnommés avec l’aide de tous les autres étudiants 
et votre aide est la bienvenue ! En effet, il y a toujours moyen de l’améliorer surtout que si le 
cours change, la synthèse doit être changée en conséquence. On peut retrouver le code source à l’adresse 
suivante
\begin{center}
	\url{https://github.com/nenglebert/Syntheses}
\end{center}\bigskip
Pour contribuer à cette synthèse, il vous suffira de créer un compte sur \textit{Github.com}. De
légères modifications (petites coquilles, orthographe, ...) peuvent directement être faites sur le
site ! Vous avez vu une petite faute ? Si oui, la corriger de cette façon ne prendra que quelques 
secondes, une bonne raison de le faire ! \bigskip

Pour de plus longues modifications, il est intéressant de disposer des fichiers : il vous 
faudra pour cela installer \LaTeX, mais aussi \textit{git}. Si cela pose problème, nous sommes 
évidemment ouverts à des contributeurs envoyant leur changement par mail ou n’importe quel autre 
moyen.\bigskip

Le lien donné ci-dessus contient aussi un \texttt{README} contenant de plus amples informations, 
vous êtes invités à le lire si vous voulez faire avancer ce projet ! 

\subsection*{Licence Creative Commons}
\begin{wrapfigure}[3]{r}{2.8cm}
	\vspace{-5mm}
	\includegraphics[scale=0.17]{../../Builder/CC}
\end{wrapfigure}
Le contenu de ce document est sous la licence Creative Commons : \textit{Attribution-NonCommercial-ShareAlike 
4.0 International (CC BY-NC-SA 4.0)}. Celle-ci vous autorise à l'exploiter pleinement, compte-
tenu de trois choses :
\begin{enumerate}
	\item \textit{Attribution} ; si vous utilisez/modifiez ce document vous devez signaler le(s) nom(s)
	      de(s) auteur(s).
	\item \textit{Non Commercial} ; interdiction de tirer un profit commercial de l’œuvre sans 
	      autorisation de l'auteur 
	\item \textit{Share alike} ;  partage de l’œuvre, avec obligation de rediffuser selon la même 
	      licence ou une licence similaire
\end{enumerate}
Si vous voulez en savoir plus sur cette licence :
\begin{center}
	\url{http://creativecommons.org/licenses/by-nc-sa/4.0/}
\end{center}

\begin{flushright}
	\textbf{Merci ! }
\end{flushright}
\tableofcontents
%Si abstract, \input ici

%%%%%%%%%%%%%%%%%%%%%
% Contenu principal %
%%%%%%%%%%%%%%%%%%%%%
\mainmatter
\chapter{Structure}
\subsection{Relation}
Étant donné deux ensemble A et B, on appelle \textit{relation} de A vers B tout ensemble de couple dont l'origine appartient à A, et l'extrêmité à B :
$$ \forall a, b \in f \subseteq A \times B : b =f(a) = l'image\ de\ A\ par\ f$$
\textbf{Vocabulaire}\\
\begin{description}
	\item[Application] Relation $A \longrightarrow B$ tel que A a pour image un seul élément de B par cette application
	\item[Injection] $\forall x,y \in A : [x \neq y \Rightarrow f(x) \neq f(y)]$
	\item[Surjection] $\forall b \in B, \exists a \in A : b = f(a)$
	\item[Bijection] $\forall b \in B, \exists a! \in A : b = f(a)$
\end{description}
On appelle relation d'équivalence, toute relation :
\begin{itemize}
	\item[Réflexive] $\forall a \in E : a \Re a$
	\item[Symétrique] $\forall a, b \in E, a \Re b = b \Re a$
	\item[Transitive] $\forall a, b, c \in E, a \Re b\ et\ b \Re c \Rightarrow a \Re c$
\end{itemize}
\ \ \\
On appelle relation d'ordre, toute relation :
\begin{itemize}
	\item[Réflexive] $\forall a \in E : a \Re a$
	\item[antisymétrique] $\forall a, b \in E, a \Re b\ et\ b \Re a \Rightarrow a = b$
	\item[Transitive] $\forall a, b, c \in E, a \Re b\ et\ b \Re c \Rightarrow a \Re c$
\end{itemize}\ \\
On dira qu'une relation d'ordre $\wp$ dans un ensemble E forme un ordre total ssi $\forall a, b \in E, a \wp b\ ou\ b \wp a$\\
\textbf{Attention :} Il n'y a pas d'ordre total dans $\mathbb{C}$.


\section{Groupe}
\subsection{Définition générale}
Etant donné un ensemble E, une loi $\clubsuit$ sur E est une application :
$$\clubsuit: E \times E \rightarrow E : (x,y) \rightarrow x\ \clubsuit\ y$$
Pour être un groupe, il faut vérifier les 4 propriétés suivantes : 
\begin{itemize}
	\item[$\clubsuit$ est \textit{interne} dans E] \ \ \ \ \ \ \ \ \ \  \ $\forall x, y \in E : x\ \clubsuit\ y \in E$
	\item[$\clubsuit$ est \textit{associative} dans E]\ \ \ \ \ \ $\forall x, y, z \in E : (x\ \clubsuit\ y)\clubsuit\ z =  x\ \clubsuit(y\ \clubsuit\ z)$
	\item[$\exists$ un \textit{neutre} pour $\clubsuit$ dans E] \ \ \  $\exists n \in E\ |\ \forall x \in E : x\ \clubsuit\ n = x = n\ \clubsuit\ x$
	\item[$\clubsuit$ est \textit{symétrisable} dans E]\ \ \ \ \ $\forall x \in E\ |\ \exists x' \in E : x\ \clubsuit\ x' = n = x'\ \clubsuit\ x$
\end{itemize}
\ \\
On dira qu'un ensemble E muni d'une loi $\clubsuit$ est un groupe \textit{commutatif} (ou \textit{abélien}) ssi :
$$E, \clubsuit\ est\ un\ groupe\ et\ [\forall x, y \in E, x\ \clubsuit\ y = y\ \clubsuit\ x$$
	\ \\
	Un groupe est \textit{d'ordre $n$} ssi $|E| = \#E = n$.\\
	Un élément $a$ d'un ensemble E muni d'une loi $\clubsuit$ est un \textit{absorbant pour $\clubsuit$} dans E ssi :
	$$a \in E\ et\ [\forall x, y \in E : x\ \clubsuit\ a = a = a\ \clubsuit\ x]$$
	\textbf{Attention :} Il y a uniticité du neutre et du symétrique (de chaque élément) pour une loi donnée dans un groupe.\\ \\
	Une autre propriété importante est la \textit{simplifiabilité} (préciser le côté !) :
	$$\forall a, b, c \in E : (a\ \clubsuit\ c) = (b\ \clubsuit\ c) \Leftrightarrow a = b\ (Simplifiabilité\ à\ droite)$$
	$$\forall a, b, c \in E : (c\ \clubsuit\ a) = (c\ \clubsuit\ b) \Leftrightarrow a = b\ (Simplifiabilité\ à\ gauche)$$
	
	\subsection{Isomorphisme de groupe}
	Deux groupes E, $\clubsuit$ et G, $\bigstar$ sont \textit{isomorphes} $\Leftrightarrow$ il existe une bijection $\delta : E \rightarrow G$ telle que : 
	$$\forall a, b \in E : \delta (a\ \clubsuit\ b) = \delta(a)\ \bigstar\ \delta(b)$$
	On dira alors que la bijection $\delta$ est un isomorphisme entre les groupes E et G.\\
	\textit{Un isomorphisme de groupe est une bijection conservant la structure du groupe}
	
	\subsection{Sous-groupes d'un groupe}
	Soit E, $\clubsuit$ un groupe. H, $\clubsuit$ est un \textit{sous-groupe} de  E, $\clubsuit$ ssi :
	\begin{itemize}
		\item[1] H est un \textit{sous-ensemble} de E
		\item[2] $\forall x, y \in H : x\ \clubsuit\ y\ \in H$
		\item[3] Le \textit{neutre} de E pour $\clubsuit \in E$
		\item[4] $\forall x \in H$ : le symétrique de $x$ pour $\clubsuit$ dans E est un élément de H
	\end{itemize}
	\ \\
	Un petit théorème en passant : \textit{Théorème de Lagrange} : Si H, $\clubsuit$ est un sous-groupe \textbf{fini} de E, $\clubsuit$ alors l'ordre de H divise l'ordre de E.
	
	\section{Anneau}
	Soit A, un ensemble munis de deux lois $\clubsuit$ et $\bigstar$.\\
	A, $\clubsuit$, $\bigstar$ est un anneau ssi :
	\begin{itemize}
		\item[1] A, $\clubsuit$ est un groupe \textit{commutatif}
		\item[2] $\bigstar$ est \textit{interne} et \textit{associatif} dans A
		\item[3] $\bigstar$ distribue $\clubsuit$ dans A
	\end{itemize}
	\ \\
	On dira que A, $\clubsuit$, $\bigstar$ est un anneau \textit{unital} ssi A, $\clubsuit$, $\bigstar$ est un anneau et s'il existe un \textit{neutre pour $\bigstar$} différent du neutre pour $\clubsuit$ dans A.
	
	\section{Corps - champ}
	Soit K, $\clubsuit$, $\bigstar$ un corps ssi : 
	\begin{itemize}
		\item[1] K, $\clubsuit$ est \textit{interne} et \textit{associatif} dans A
		\item[2] $K_{n}$, $bigstar$ est un groupe \textit{commutatif} (ou $n$ est neutre pour $\clubsuit$ dans K)
		      
		\item[3] $\bigstar$ distribue $\clubsuit$ dans A
	\end{itemize}
	\ \\
	
	Soit K, un ensemble muni de deux lois $\clubsuit$ et  $\bigstar$.\\
	K, $\clubsuit$, $\bigstar$ un corps commutatif ou \textit{champ} ssi :
	\begin{itemize}
		\item[1] K, $\clubsuit$, $\bigstar$ est un corps
		\item[2] $\bigstar$ est \textit{commutative} dans K
	\end{itemize}
	
	\chapter{Espaces Vectoriels}
	\section{Définition}
	A INCLURE.
	
	\section{Exemples d'EV}
	\textit{Cf. cours}
	
	\section{Prorpiétés des espaces vectoriels}
	Si K, V, + est un espace vectoriel, alors :
	\begin{enumerate}
		\item $\forall x \in V : 0.\vec{x} = \vec{0}$
		\item $\forall \lambda \in K : \lambda.\vec{0} = \vec{0}$
		\item $\forall \lambda \in K, \forall \vec{x} \in V : \lambda \vec{0} \Rightarrow \lambda = 0\ ou\ \vec{x} = \vec{0}$
		\item $\forall \lambda \in K, \forall \vec{x} \in V : \lambda(- \vec{x}) = -\lambda\vec{x}$
		\item $\forall \lambda_{i} \in K, \forall \vec{x} \in V : (\sum_{1}^{i} \lambda_{i})\vec{x} = \sum_{1}^{i}(\lambda_{i}\vec{x})$
		\item $\forall \lambda_{i} \in K, \forall \vec{x} \in V : (\sum_{1}^{i} \vec{x}_{i})\lambda = \sum_{1}^{i}(\lambda\vec{x_{i}})$
	\end{enumerate}
	
	\section{Sous-espaces vectoriels}
	Soit W, un sous-ensemble d'un espace vectoriel K, V, + défini sur un corps K\\
	K, W, + est un \textit{sous-espace vectoriel} ssi :
	\begin{enumerate}
		\item W est non vide (signifie $\vec{0} \in W$)
		\item $\forall \vec{x}, \vec{y} \in W = \vec{x} + \vec{y} \in W$
		\item $\forall \vec{x} \in W, \forall \lambda \in K : \lambda\vec{x} \in W$
	\end{enumerate}
	\textbf{Attention :} Les lois et corps doivent être identiques !\\
	\textit{NB :} Un espace vectoriel est le plus grand sous-vectoriel de lui-même. De même K, $\{\vec{0}\}$, + est le plus petit des EV. Ces deux SV sont dit \textit{triviaux}.\\
	\textit{NB.2 } L'intersection de deux sous-espaces vectoriel et un SEV.
	
	\subsection{Lien avec les équations linéaire homogènes}
	L'ensemble des solution d'une équation linéaire homogène à $n$ inconnues et à coefficient dans un corps K est un sous-espace vectoriel de $K, K^{n}, +$.
	
	\subsection{Lien avec les systèmes d'équations linéaires homogènes}
	L'ensemble des solution d'un système d'équations linéaires homogènes est un sous-espace vectoriel de $K^{n}$ défini sur un corps K (commutatif).
	
	\section{Somme de SEV}
	Soit K, V, + un espace vectoriel et $W_{1}, W_{2}$ deux SEV de V. On appelle \textit{somme de $W_{1}, W_{2}$} l'ensemble défini par :
	$$W_{1}, W_{2} = \{\vec{w_{1}} + \vec{w_{2}}\ |\ \vec{w_{1}} \in W_{1}, \vec{w_{2}} \in W_{2}\}$$
	
	Soit V, + un  EV sur un corps K et $W_{1}, W_{2}$ deux SEV de V. On dira que V est la somme directe (notée $\bigoplus$) de $W_{1}, W_{2}$ ssi
	$$[V = W_{1} + W_{2}\ et\ W_{1} \cap W_{2} = \{\vec{0}\}]$$
	
	\section{Isomorphisme d'espaces vectoriels}
	Soient K, V, + et K, W, + deux EV défini sur le \textit{même corps}.\\
	V est \textit{isomorphe} ) W (V $\cong $ W) ssi il existe une bijection $\sigma V \rightarrow W$ telle que : 
	\begin{enumerate}
		\item $\forall \vec{x}, \vec{y} \in V : \sigma(\vec{x} + \vec{y}) = \sigma(\vec{x}) + \sigma(\vec{y})$
		\item $\forall \vec{x} \in V, \forall \lambda \in K : \sigma(\lambda\vec{x}) = \lambda\sigma(\vec{x})$
	\end{enumerate}
	\textbf{Attention :} Il s'agit d'une question typique de la partie théorique de l'examen de janvier.
	
	\section{Parties génératrices}
	Si K, V, + est un espace vectoriel et si P est un sous-ensemble de V, alors :\\
	\textit{L(P)} est le sous-espace de V, engendré par P.\\\\
	\textit{NB :} Toute partie contenant une partie génératrice est une partie génératrice. Une partie génératrice est dite minimale, s'il n'existe pas de sous ensemble inclus dans P qui soit également génératrice.
	
	\section{Combinaisons linéaires de vecteurs}
	Si X est une partie d'un espace vectoriel K, V, +, on appelle \textit{combinaison linéaire des vecteurs de X} ou \textit{combili des vecteurs de X} tout vecteur V de la forme :
	$$\lambda_{1}\vec{x_{1}} + \lambda_{2}\vec{x_{2}} + ... + \lambda_{n}\vec{x_{n}}$$
	où les $\vec{x_{i}}$ sont les éléments de X \textbf{en nombre fini} et $\lambda_{i}$ sont des élément du corps K.\\\\
	\textit{Théorème :} Pour toute partie non vide X d'un espace vectoriel K, V, + le sous-espace L(X) engendré par X est l'ensemble des combili des vecteurs de X.
	
	\section{Parties libres de vecteurs}
	\textit{Voir horrible définition dans le cours (\textbf{Attention :} AVC possible)}.\\
	Notons tout de même que si X est un sous-ensemble d'un EV K, V, + contenant $\vec{0}$ alors X \textbf{n'est pas} une partie libre de V.\\
	\textit{NB :} Si X est une partie libre d'un EV, alors tout sous-ensemble de X est une partie libre de V (L'ensemble vide également)\\\\
	\textbf{Théorème } : Soit X, une partie libre d'un EV K, V, +.\\
	X est une PL ssi :
	$$\forall \vec{x_{i}} \in X, \forall \lambda_{i} \in K$$
	$$[\sum_{i=1}^{n} \lambda_{i}\vec{x_{i}} = \vec{0} \Rightarrow \lambda_{1}, \lambda_{2}, ... \lambda_{n} = 0]$$
	Deux vecteurs sont LI s'ils forment une PL de V.
	
	\section{Bases et dimension}
	Une partie sera une base $\Leftrightarrow$ celle-ci est à la fois libre et génératrice. On dira qu'une base est une \textit{partie libre minimale} et une \textit{partie génératrice maximale}.\\\\
	\textbf{Théorème :} Si K, V, + est un espace vectoriel, si L est une PL de V et si G est une PG de V contenant L, alors il existe une base B de V telle que $L \subseteq B \subseteq G$.\\
	\textit{Ce théorème permet d'étendre une partie libre pour former une base}.\\\\
	Notons également :
	\begin{itemize}
		\item Tout EV possède une base.
		\item S'il existe une base finie de n élément, toute base comporte n élément.
		\item Les bases d'un même EV ont le même cardinal.
		\item Toute PL a au plus n éléments.
		\item Toute PG a au max n éléments.
	\end{itemize}
	\ \\
	Si la dimension d'un EV est finie et vaut n, alors :
	\begin{itemize}
		\item Toute partie libre de n éléments est une base de cet EV.
		\item Toute partie génératrice de n éléments est une base de cet EV.
	\end{itemize}
	\ \\
	\textit{NB :} $dim(W_{1} + W_{2}) = dim(W_{1}) + dim(W_{2}) - dim(W_{1} \cap W_{2})$\\\\
	Si B est une base d'un espace vectoriel, alors tout vecteur de V s'exprime \textbf{d'une et une seule manière} comme combili d'un nombre \textbf{fini} des vecteurs de B.
	
	\section{Base canonique}
	Liste de bases à connaître par coeur ! \textit{Cf. cours}
	
	\section{Écriture matricielle d'une vecteur dans une base donnée}
	$\forall x \in V : \exists 1!(\lambda_{1}, \lambda_{2}, ..., \lambda_{n}) \in K^{n}\ |\ \vec{x} = \sum_{i=1}^{n} \lambda_{i}\vec{x_{i}}$, les vecteurs $\vec{e_{i}}$ étant vecteurs de B.\\
	\textbf{Convention de notation :}\\
	$$\vec{x} = \sum_{i=1}^{n}  \lambda_{i}\vec{e_{i}} = \lambda_{1}\vec{e_{1}}, \lambda_{2}\vec{e_{2}}, ..., \lambda_{n}\vec{e_{n}} = (\vec{e_{1}}\ \vec{e_{2}}\ \ \ \ \vec{e_{n}})\begin{pmatrix} 
	\lambda_1 \\ 
	\lambda_2 \\ 
	\vdots\\ 
	\lambda_2 \end{pmatrix}$$
	On nommera $\begin{pmatrix} 
	\lambda_1 \\ 
	\lambda_2 \\ 
	\vdots\\ 
	\lambda_2 \end{pmatrix}$ \textit{matrice des composantes de $\vec{x}$ dans la base B. Notation : $X_B$}.
	\\
	\textbf{Attention :} Ne pas convondre un vecteur et sa matrice de 
	composantes.
	
	
	\section{Changement de bases et de composantes}
	(\textit{On suppose K, V, + est un EV de dimension finie sur un corps K commutatif})\\
	\subsection{Matrice de changement de base de $e$ vers $\epsilon$}
	Cette section (et les deux suivantes) étant principalement pratique, je ne me contenterai ici que de reprendre les notations (\textit{cf. TP5/6} :) )\\
	$P_e^a$ se lit :
	\begin{itemize}
		\item Matrice de changement de base de $e$ vers $a$ (On "monte" dans les \textbf{B}ase comme \textbf{B}uzz l'éclair).
		\item Matrice de changement de composante de $a$ vers $e$ (On "descend", on \textbf{C}reuse les \textbf{C}omposantes).
	\end{itemize}
	\subsection{Détermination de la patrice $P_\epsilon^e$}
	$$P_\epsilon^e = (P_e^\epsilon)^{-1}$$
	
	\subsection{Détermination de $X_\epsilon$ des vecteurs de $\vec{x}$ dans la base $\epsilon$ à partir de la matrice $X_e$ des composantes de $\vec{x}$ dans la base $e$}
	$$X_\epsilon = P_\epsilon^e X_e$$
	où $P_\epsilon^e$ est la matrice de changement de base de $\epsilon$ vers $e$.
	
	\section{Droites, plans et hyperplans vectoriels}
	\subsection{Droite vectorielle}
	Si K, V, + est un espace vectoriel de dimension au moins 1, alors on appelle une \textit{droite (vectorielle)} de V tout SEV de V de dimension 1.
	
	\subsection{Plan vectoriel}
	Si K, V, + est un espace vectoriel de dimension au moins 2, alors on appelle une \textit{plan (vectoriel)} de V tout SEV de V de dimension 2.
	
	\subsection{Hyperplan vectoriel}
	Soit K, V, + un espace vectoriel. Un sous-espace vectoriel est un \textit{hyperplan (vectoriel)} de V $\Leftrightarrow$ H est un sous-espace vectoriel propre et maximal de V.\\
	\begin{itemize}
		\item[Maximal] : $\Leftrightarrow \nexists$ de SEV S de V tel que $H\subset S \subset V$
		\item[Propre] : $H \neq V$
	\end{itemize}
	\ \\
	\textit{La fin du chapitre est à lire à titre informatif}.
	
	
	\chapter{Applications linéaires}
	\section{Définition}
	Si K, V, + et K, W, + sont deux espaces vectoriels définis sur un même corps K, on appelle \textit{application linéaire} de V dans W tout application $\sigma : V \rightarrow W$ telle que :
	$$\forall \vec{x}, \vec{y} \in V : \sigma(\vec{x} + \vec{y}) = \sigma(\vec{x}) + \sigma(\vec{y})$$
	$$\forall \lambda \in K, \forall \vec{x} \in V : \sigma(\lambda\vec{x}) = \lambda \sigma(\vec{x})$$
	\textbf{Cas particuliers importants :} 
	\begin{enumerate}
		\item Si $\sigma$ est une application linéaires de V dans W, avec V=W, alors $\sigma$ es appelé \textit{opérateur linéaire} ou \textit{endomorphisme} de V.
		\item Si $\sigma$ est une application de V dans W, avec W=K, alors $\sigma$ est une \textit{forme linéaire} ou \textit{covecteur} définie sur V.
		\item Si $\sigma$ est une bijection linéaire de V dans W, alors $\sigma$ est un \textit{isomorphisme} de V sur W.
		\item Si $\sigma$ est un opérateur linéaire de V \textit{et} une bijection, alors $\sigma$ est un \textit{automorphisme} de V.
	\end{enumerate}
	
	\section{Matrices d'une AL dans les bases données}
	
	Encore une fois, c'est principalement pratique (\textit{cf TP6}). Néanmoins : \\
	$A_\epsilon^e$, matrice de $\alpha$ dans les bases $e$ et $\epsilon$, est une matrice de p ligne(cardinal de la base $\epsilon$ et de n colonnes (cardinal de la base $e$).\\\\
	\textbf{Attention :} Les matrices de rotations sont à étudier par coeur! Ne pas oublier de donner les bases, sinon cela n'a pas de sens ! 
	
	\section{Composante de l'image d'un vecteur par une AL}
	Même remarque qu'au point précédent, partie essentiellement pratique.
	
	\section{Composée de deux AL}
	Si K, V, +, K, W, + et K, S, + sont trois espaces vectoriels définis sur un même corps K et si $\alpha : V \rightarrow W$ et $\beta : W \rightarrow S$ sont deux applications linéaires, alors $\beta  \circ  \alpha$ est une application linéaire.
	
	\section{Matrices d'AL et changement de base}
	Partie pratique. (\textit{Cf. TP6})
	
	\section{Espace vectoriel des AL}
	\subsection{Théorème 1}
	Si K, V, + et K, W, + sont deux espaces vectoriels définis sur un coprs K \textit{commutatif} et si $L(V, W)$ est l'ensemble de toutes les applications linéaire $V \rightarrow W$, alors $K, L(V,W), +$ est un espace vectoriel sur K.\\\\
	\textbf{Attention :} Regarder attentivement la définition de l'addition vectorielle et de la multiplication par un scalaire.
	
	\subsection{Théorème 4}
	Si K, V, + est un espace vectoriel de dimension \textit{finie} $n$, défini sur un corps K \textit{commutatif}, et si K, W, + est un espace vectoriel de dimension \textit{finie} p défini sur le même corps K, alors :
	$$K, L(V, W), + \cong K, K^{p \times n}, +$$
	
	\textbf{Corolaire}
	$$dim(L(V,W)) = pn$$
	
	\section{Noyeau d'une AL}
	Si K, V, + et K, W, + sont deux espaces vectoriels définis sur un corps K \textit{commutatif}, et si $\alpha : V \rightarrow W$ est une application linéaire, alors le \textbf{noyau de $\alpha$}, noté \textit{ker $\alpha$}, est l'ensemble :
	$$ker\ \alpha = \{\vec{x} \in V\ |\ \alpha(\vec{x}) = \vec{0}\ (de\ W)\}$$
	
	\subsection{Théorème 1}
	Si K, V, + et K, W, + sont deux espaces vectoriels définis sur un corps K commutatif et si $\alpha V \rightarrow W$ est une application linéaire, alors le vecteur nul de V appartient toujours au noyau de $\alpha$.
	
	\subsection{Théorème 2}
	Si K, V, + et K, W, + sont deux espaces vectoriels définis sur un corps K commutatif, et si $\alpha : V \rightarrow W$ est une AL, alors :
	$$\alpha\ est\ injective\ \Leftrightarrow ker\ \alpha = \{\vec{0}\}$$
	
	\subsection{Théorème 3}
	Si K, V, + et K, W, + sont deux espaces vectoriels définis sur un corps K,  et si $\alpha : V \rightarrow W$ est une AL, alors K, $ker\ \alpha$, + est un sous-espace vectoriel de K, V, +
	
	\subsection{Autre définition}
	Si K, V, + et K, W, + sont deux espaces vectoriels définis sur un corps K,  et si $\alpha : V \rightarrow W$ est une AL, alors on appelle \textit{nullité de $\alpha$} la dimension de $\ker\ \alpha$.
	
	\section{Image d'une AL}
	Si K, V, + et K, W, + sont deux espaces vectoriels définis sur un corps K \textit{commutatif},  et si $\alpha : V \rightarrow W$ est une AL, alors on appelle \textit{image de $\alpha$}, noté $Im\ \alpha$ l'ensemble :
	$$Im\ \alpha = \{\vec{y} \in W\ |\ \exists \vec{x} \in V\ avec\ \alpha(\vec{x}) = \vec{y}\}$$
	
	\subsection{Théorème 4}
	Si K, V, + et K, W, + sont deux espaces vectoriels définis sur un corps K,  et si $\alpha : V \rightarrow W$ est une AL alors :
	$$\alpha\ est\ surjective\ \Leftrightarrow\ Im\ \alpha = W$$
	
	\textbf{Corolaire :}\\
	Si K, V, + et K, W, + sont deux espaces vectoriels définis sur un corps K,  et si $\alpha : V \rightarrow W$ est une AL alors $\alpha$ est \textit{bijective} $\Leftrightarrow$
	$$ker\ \alpha = \{\vec{0}\}\ et\ Im\ \alpha = W$$
	
	\subsection{Théorème 5}
	Si K, V, + et K, W, + sont deux espaces vectoriels définis sur un corps K, alors l'image d'un sous-espace vectoriel $S$ de $V$ par une AL$\alpha : V \rightarrow W$ est un sous-espace vectoriel de $W$.
	\\\\
	\textbf{Corolaire :}\\
	Si K, V, + et K, W, + sont deux espaces vectoriels définis sur un corps K,  et si $\alpha : V \rightarrow W$ est une AL, alors K, Im $\alpha$, + est un SEV de K, W, +.
	
	\subsection{Autre définition}
	Si K, V, + et K, W, + sont deux espaces vectoriels définis sur un corps K,  et si $\alpha : V \rightarrow W$ est une AL, alors on appelle \textit{rang de $\alpha$}, la dimension de Im $\alpha$.
	
	\subsection{Théorème 6 (Très important)}
	Si K, V, + et K, W, + sont deux espaces vectoriels définis sur un corps K,  et si $\alpha : V \rightarrow W$ est une AL, et si K, V, + est de dimension \textit{finie} $n$ alors :
	\begin{center}
		L'image d'une base par $\alpha$ est une partie génératrice de Im $\alpha$.
	\end{center}
	
	\section{Lien entre noyau et image d'une AL}
	\subsection{Théorème 7 (Fondamental)}
	Si K, V, + et K, W, + sont deux espaces vectoriels définis sur un corps K,  et si $\sigma : V \rightarrow W$ est une AL,et si V est de dimension \textit{finie} $n$, alors :
	\begin{center}
		dim(ker $\sigma$) + dim(Im $\sigma$) = dim V
	\end{center}
	
	\section{Formes linéaires - Espace dual}
	\subsection{Définitions et propriétés}
	Si K, V, + est un espace vectoriel défini sur un corps K commutatif, alors l'ensemble des formes linéaires de V dans K est appelé l'\textit{espace dual} de V et est noté $V^{*}$.\\
	Autrement dit : $V^* = L(V, K)$.
	
	\subsection{Exemples de formes linéaires}
	\textit{Cf. cours}
	
	\subsection{Forme linéaire et noyau}
	\textbf{Théorème 1}\\
	Si K, V, + est un espace vectoriel défini sur un corps K commutatif et si $f : V \rightarrow K$ est une forme linéaire sur V, alors :
	\begin{center}
		Soit ker($f$) est un hyperplan de V, soit ker($f$) = V
	\end{center}
	\ \\
	\textbf{Défintion 3 :}\\
	Si K, V, + est un espace vectoriel défini sur un corps K commutatif. Une forme linéaire $f : V \rightarrow K$ est \textit{non dégénérée} $\Leftrightarrow$
	$$ker(f)\ est\ un\ hyperplan\ de\ V$$
	La forme linéaire sera donc \textit{dégénérée} $\Leftrightarrow ker(f) = V$.
	\subsection{Interprétation géométrique des forme linéaires}
	Ne fait pas partie de l'examen.
	
	\subsection{Expression d'une forme linéaire par rapport à une base de V}
	Si K, V, + est un espace vectoriel de dimension finie défini sur un corps K commutatif, alors toute forme linéaire $f : V \rightarrow L$ est \textit{univoquement déterminée} par les valeurs qu'elle prend sur les éléments d'une base de V.
	
	\subsection{Base de $K, V^*, +$ en dimension finie}
	Si K, V, + est un espace vectoriel de dimension finie $n$ défini sur un corps K commutatif, et muni d'une base $e = (\vec{e_1}, \vec{e_2}, ..., \vec{e_n})$ alors les $n$ applications :
	$$e_i^* : V \rightarrow K : \vec{x} = \sum_{j=1}^{n} x_j \vec{e_j} \rightarrow x_i = i^{ème}\ composante\ de\ \vec{x}\ dans\ la\ base\ e$$
	forment une base de $K, V^*, +$, appelée \textit{base duale de la base e} qui sera notée $e^* = (\vec{e^*_1}, \vec{e^*_2}, ..., \vec{e^*_n})$\\
	\ \\
	\textbf{Propriété 1 (essentielle)}\\
	Si K, V, + est un espace vectoriel de dimension finie $n$ défini sur un corps K commutatif, et muni d'une base $e = (\vec{e_1}, \vec{e_2}, ..., \vec{e_n})$ et si $e^* = (\vec{e^*_1}, \vec{e^*_2}, ..., \vec{e^*_n})$, alors :
	$$\forall i, j = 1, ..., n = e_i^*(\vec{e_j}) = \delta_{ij}$$
	(où $\delta_{ij}$ est le symbole de Kronecker)
	
	\subsection{Composante d'une forme linéaire dans une base de $K, V^*, +$}
	Partie essentiellement pratique (\textit{cf. TP}).
	
	
	\chapter{Rang de matrices - Systèmes d'équations linéaires}
	\section{Rang d'une matrice}
	Cette partie est fort \textit{recopiage} mais il n'y a pas vraiment le choix.
	\subsection{Rang d'un système de vecteurs et rang d'une matrice}
	\textbf{Définition 1} :\\
	Le \textit{rang d'un système de vecteurs} d'un espace vectoriel K, V, + est la dimension du sous-espace vectoriel de $V$, engendré par cet ensemble de vecteurs.\\
	\\
	\textbf{Définition 2} : \\
	Si $A$ est une matrice d'ordre $p \times n$  à coefficients dans un corps K commutatif,alors le \textit{rang de A} est le nombre maximale de \textit{"colonnes de A linéairement indépendantes"}, c'est-à-dire de manière plus précise que le \textit{rang de A} est le rang du système de vecteurs-colonnes déterminé par les $n$ colonnes de $A$.\\
	\\
	\textit{NB :} Plus simplement (et moins mathématiquement du coup \ ;D), on peut considérer le le rang de A est le nombre de \textit{colonnes} linéairement indépendantes.
	
	\subsection{Détermination du rang d'une matrice A d'ordre (p,n)}
	C'est long et peu utile ici, regardez les slides ou elle donne un bon gros exemple bien détaillé !
	
	\subsection{Propriété des rangs de matrices}
	\textit{Si A est une matrice d'ordre $p \times n$} à coefficients dans un corps K commutatif, alors  :
	$$rang(A) \leq n$$
	\textit{Si A est une matrice d'ordre $p \times n$} à coefficients dans un corps K commutatif, alors :
	$$rang(A) = rang( ^t A)$$
	\textbf{Corollaire 1 :}\\
	i $A$ est une matrice d'ordre $p \times n$  à coefficients dans un corps K commutatif,alors :
	$$rang(A) \leq min{p,n}$$
	C'est 'logique' dans le sens ou une matrice de trois colonnes ne peut avoir quatre (et plus ) vecteurs LI.\\
	\\
	\textbf{Corrolaire 2 :}\\
	Le rang d'une matrice $A$ ) coefficients dans un corps K commutatif peut se calculer aussi bien à partir d'un système de vecteurs-ligne associé à cette matrice qu'à partir du système de vecteurs-colonne associé.\\
	\\
	\textbf{Théorème :}\\
	Si $A$ est une matrice d'ordre $p \times n$  à coefficients dans un corps K commutatif,alors :
	$$rang(A) = dim(Im \alpha)$$
	ou $\alpha : V \rightarrow W$ est une application linéaire, K, V, + est une EV de dimension $n$ et de base $e$, K, W, + est une EV de dimension $p$ et de base $\epsilon$ et la matrice $\alpha$ dans les bases $e$ et $\epsilon$ est la matrice donnée $A$.\\
	\\
	\textbf{Lien avec les déterminants :}
	Si $A$ est une matrice d'ordre $p \times n$  à coefficients dans un corps K commutatif. Considérons l'ensemble $\lbrace A_i\ \vert\ i=1,...,min\{p,n\}\rbrace$ de toutes les sous-matrices carrées extraites de $A$. On peut alors démontrer que :
	$$rang(A) = max\ \lbrace\ ordre\ A_i\ \vert\ det(A_i) \neq 0 ; i = 1,...,min\{p,n\}\ \rbrace$$
	\\
	En français abusif, on peut dire que : '\textit{le rang de A est l'ordre du plus gros déterminant non nul que l'on peut extraire de la matrice A}'.\\\\
	\textit{NB :} C'est généralement une mauvaise idée de faire ça, il faut plus l'utiliser quand on est presque à la fin si ça peut faire gagner du temps.
	
	
	\chapter{Formes bilinéaires et produits scalaires}
	\section{Formes bilinéaires}
	\subsection{Définitions}
	Si V, + est un espace vectoriel réel de dimension finie, sur un corps K commutatif, on appelle \textbf{forme bilinéaire sur V} toute application 
	$$f : V \times V \rightarrow K : (\vec{x}, \vec{y}) \rightarrow f(\vec{x}, \vec{y})$$
	tel qu'elle est linéaire à gauche et à droite\\
	
	On dira qu'une forme bilinéaire $f : V \times V \rightarrow K$ est \textbf{symétrique} $\Leftrightarrow \forall \vec{x}, \vec{y} \in V : f(\vec{x}, \vec{y}) = f(\vec{y}, \vec{x})$.
	
	\subsection{Exemples de formes bilinéaires}
	\textbf{Produit scalaire usuel}\\
	Soit $V = \underline{\mathbb{R}^n}$, on appelle produit scalaire usuel l'application :
	$$< , > : \mathbb{R}^n \times \mathbb{R}^n \rightarrow \mathbb{R} = (\vec{x}, \vec{y}) \rightarrow <\vec{x}, \vec{y}> = \sum_{i=1}^n x_iy_i$$
	avec $\vec{x} = (x_1, x_2, ..., x_n) \in \mathbb{R}^n$ et $\vec{y} = (y_1, y_2, ..., y_n) \in \mathbb{R}^n$.
	
	\textbf{Attention :} il ne nécessite aucune base, les vecteurs sont des n-uples réels.\\
	
	\emph{Théorème} : 
	Le produit salaire usuel est une forme bilinéaire \textbf{symétrique} sur le corps des réels.
	
	\section{Espace vectoriel des formes bilinéaires}
	Soit \textit{Bil}(V) \textbf{l'ensemble des formes bilinéaires} sur un espace vectoriel K,V,+. \\
	Par conséquent :
	$f \in \mathcal{B}il(V)$ signifie que $f : V \times V \rightarrow K : (\vec{x}, \vec{y}) \rightarrow f(\vec{x}, \vec{y})$\\
	Si on munit $\mathcal{B}il(V)$ d'une addition vectorielle et d'une multiplication par un scalaire (\textit{def p. 161}) alors $K,\mathcal{B}il(V),+$ est un espace vectoriel.
	
	\subsection*{Théorème}
	Si V, + est un espace vectoriel de dimension finie n sur un corps K commutatif et si $e = (\vec{e_1}, ..., \vec{e_n})$ est une base de V, alors les $n^2$ formes bilinéaires 
	$$g_{ij} : V \times V \rightarrow K : (\vec{x}, \vec{y}) \rightarrow g_{ij}(\vec{x}, \vec{y}) = e_i^*(\vec{x})e_j^*(\vec{y})\ (\forall i, j : 1,2,...n)$$
	forment une base de $K,\mathcal{B}il(V),+$.\\
	
	\emph{Corolaire}\\
	Soient V, + est un espace vectoriel de dimension finie n sur un corps K commutatif muni d'une base $e = (\vec{e_1}, ..., \vec{e_n})$ et $f : V \times V \rightarrow K : (\vec{x}, \vec{y}) \rightarrow f(\vec{x}, \vec{y})$ une forme bilinéaire définie sur V, alors
	$$f = \sum_{i=1}^n\sum_{j=1}^n f(\vec{e_i}\vec{e_j})g_{ij}$$
	et les composantes de $f$ dans la base $(j_{ij} | i,j= 1,2,...n)$ de \textit{Bil}(V) sont \textbf{les images par f des couples de vecteurs de la base e}.(\textit{Bon exemple page 165}).
	
	\section{Matrice d'une forme bilinéaire dans une base donnée}
	Pour aller droit au but, en considérant la base canonique $e = (\vec{e_1}, \vec{e_2}, \vec{e_3})$, \textbf{la matrice $F^e$ de la forme bilinéaire $f$ dans la base $e$} :
	$$\begin{pmatrix}
	f(\vec{e_1}, \vec{e_1}) & f(\vec{e_1}, \vec{e_2}) & f(\vec{e_1}, \vec{e_3})\\
	f(\vec{e_2}, \vec{e_1}) & f(\vec{e_2}, \vec{e_2}) & f(\vec{e_2}, \vec{e_3})\\
	f(\vec{e_3}, \vec{e_1}) & f(\vec{e_3}, \vec{e_2}) & f(\vec{e_3}, \vec{e_3})\\
	\end{pmatrix}$$
	Connaissant deux vecteurs $\vec{x}$ et $\vec{y}$, on peut calculer $f(\vec{x}, \vec{y})$ de deux façons différentes : en remplaçant directement dans l'expression ou de façon matricielle :$f(\vec{x}, \vec{y}) =\ ^tX_e F^e Y_e$.
	$$f(\vec{x}, \vec{y})\ =\ \left(x_1, x_2, x_3\right)\begin{pmatrix}
	f(\vec{e_1}, \vec{e_1}) & f(\vec{e_1}, \vec{e_2}) & f(\vec{e_1}, \vec{e_3})\\
	f(\vec{e_2}, \vec{e_1}) & f(\vec{e_2}, \vec{e_2}) & f(\vec{e_2}, \vec{e_3})\\
	f(\vec{e_3}, \vec{e_1}) & f(\vec{e_3}, \vec{e_2}) & f(\vec{e_3}, \vec{e_3})\\
	\end{pmatrix}\begin{pmatrix}
	y_1\\
	y_2\\
	y_3\\
	\end{pmatrix}$$
	
	\section{Formes bilinéaires et changements de bases}
	Le lien entre la matrice $F^e$ de $f$ dans la base $e$ et la matrice $F^u$ de $f$ dans la base $uu$ est :
	$$F^u =\ ^t(P_e^u)F^eP^u_e$$
	\emph{Définition}\\
	Le \textbf{rang d'une forme bilinéaire} $f : V \times V \rightarrow K$ est le rang de la matrice de $f$ dans une base quelconque de V.
	
	\section{Produits scalaires}
	\subsection{Définitions}
	\emph{Définition 1}\\
	Si V, + est un espace vectoriel \textbf{réel} on appelle \textbf{produit scalaire sur V} tout \textit{forme bilinéaire symétrique} sur V. On utilisera pour les produits scalaire la notation qui suit :
	$$< , > : V \times V \rightarrow \mathbb{R} = (\vec{x}, \vec{y}) \rightarrow <\vec{x}, \vec{y}>$$\\
	
	\emph{Définition 2}\\
	Un produit scalaire $< , >$ défini sur un espace vectoriel $\mathbb{R}, V, +$ est \textbf{défini positif} $\Leftrightarrow\  < , >$ est un produit scalaire tel que :
	\begin{enumerate}
		\item $\forall \vec{x} \in V : <\vec{x}, \vec{x}> \geq 0\ \ (positif)$
		\item $< \vec{x}, \vec{x}> = 0 \Leftrightarrow \vec{x} = \vec{0}$
	\end{enumerate}
	\subsection{Exemples de P.Scal (à connaître)}
	\begin{enumerate}
		\item Le produit scalaire \textbf{usuel} (voir plus haut)
		\item $< , > : \mathcal{P}_3 \times \mathcal{P}_3 \rightarrow \mathbb{R} : (\vec{p}, \vec{q}) \rightarrow <\vec{p}, \vec{q}> = \int_0^1 p(t)q(t) dt$
		\item $< , > : \mathbb{R}^{p \times n} \times \mathbb{R}^{p \times n} \rightarrow \mathbb{R} : (A, B) \rightarrow <A, B> = tr(A\ \ ^tB)$
	\end{enumerate}
	
	\section{Espaces euclidiens centrés}
	Un espace \textbf{euclidien centré} est un espace vectoriel \textbf{réel} muni d'un produit scalaire \textbf{défini positif} noté $\mathbb{R},V,+, < , >$.
	
	\section{Propriété de formes bilinéaires particulières}
	\textbf{Définition 1} : Une forme bilinéaire $f : V \times V \rightarrow K$ est \textbf{symétrique} ssi $\forall \vec{x}, \vec{y} \in V = f(\vec{x}, \vec{y}) = f(\vec{y}, \vec{x})$.\\
	
	\textbf{Théorème 1} :
	Une forme bilinéaire $f : V \times V \rightarrow K$ est symétrique ssi il existe une base $e$ de $V$ dans laquelle la matrice de $f$ est symétrique.\\
	
	\textbf{Définition 2} : Une forme bilinéaire $f : V \times V \rightarrow K$ est \textbf{antisymétrique} ssi $\forall \vec{x}, \vec{y} \in V = f(\vec{x}, \vec{y}) = -f(\vec{y}, \vec{x})$.\\
	
	\textbf{Définition 3} : Une forme bilinéaire $f : V \times V \rightarrow K$ est \textbf{alternée} ssi $\forall \vec{x} \in V : f(\vec{x}, \vec{x}) = 0$.\\
	
	\textbf{Théorème 2} : Si $2 \neq 0$ dans K, alors tout forme bilinéaire antisymétrique est alternée.\\
	
	\textbf{Théorème 3} : Toute forme bilinéaire alternée est antisymétrique.\\
	
	\textbf{Définition 1} : Si $2 \neq 0$ dans K, alors toute forme bilinéaire $f : V \times V \rightarrow K$ est la somme d'une forme bilinéaire symétrique et d'une forme bilinéaire alternée.\\
	
	\section{Orthogonalité}
	\subsection*{Définition}
	Soit K, V, + un EV de dimension finie, défini sur un corps K commutatif et $f : V \times V \rightarrow K$ une forme bilinéaire sur $V$. On dira qu'un \textbf{vecteur} $\vec{x} \in V$ est \textbf{orthogonal} à $\vec{y} \in V\ \ (\vec{x} \perp \vec{y}) \Leftrightarrow$
	$$f(\vec{x}, \vec{y}) = 0$$
	Hélas, $\perp$ n'est pas toujours symétrique : si $f(\vec{x}, \vec{y}) 0 = 0 \neq f(\vec{y},\vec{x})$ on risque d'avoir un souci ! Sont-ils perpendiculaires ou non ? Pour régler le souci, il y a la \textit{réflexivité}.
	
	\subsection*{Définition}
	Une forme bilinéaire $f : V \times V \rightarrow K$ est \textbf{réflexive} 
	$$\Leftrightarrow \forall \vec{x}, \vec{y} \in V \left[ \vec{x} \perp \vec{y} \Rightarrow \vec{y} \perp \vec{x}\right]$$
	Ainsi, une forme bilinéaire est réflexive $\Leftrightarrow$ \textbf{f est symétrique ou alternée}.\\
	
	Hélas, il peut exister des vecteurs non nuls de $V$ qui sont orthogonaux à eux mêmes.
	\subsection*{Définition}
	Soit K, V, + un EV de dimension finie, défini sur un corps K commutatif et $f : V \times V \rightarrow K$ une forme bilinéaire sur $V$. \\
	On appelle \textbf{vecteur isotrope de V}, tout vecteur $\vec{x} \in V$ tel que $f(\vec{x}, \vec{y}) = 0$.\\
	
	Pire, s'il existe au moins un vecteur non nul de $V$ qui est orthogonal à tous les vecteurs de $V$ on dira que $f$ est \textbf{dégénérée}. Dans le cas contraire (ou il n'y en a aucun) f est dite \textbf{non dégénérée}.
	
	\subsection*{Théorème}
	En passant le blabla de la base et de l'EV, si $F^e$ est la matrice d'une forme bilinéaire dans la base $e$ alors f est \textbf{dégénérée} 
	$$\Leftrightarrow det(F^e) = 0\ \ \Leftrightarrow rang\ de\ f\ < n$$
	
	
	\chapter{Produits hermitiens}
	\section{Produits hermitiens ou forme hermitiennes}
	Si $V, +$ est un espace vectoriel \underline{complexe} ($K = \mathbb{C}$) de dimension finie, on appelle \textbf{produit hermitien} ou \textbf{forme hermitienne sur V} tout application $h : V \times V \rightarrow \mathbb{C} : (\vec{x}, \vec{y}) \rightarrow h(\vec{x}, \vec{y})$ telle que $\forall \vec{x}, \vec{y}, \vec{z} \in V\ et\ \forall \lambda \in \mathbb{C} :$
	\begin{enumerate}
		\item $h(\vec{x} + \vec{y}, \vec{z}) = h(\vec{x}, \vec{z}) + h(\vec{y}, \vec{z})$
		\item $h(\lambda\vec{x}, \vec{z}) = \lambda h(\vec{x}, \vec{z})$
		\item $h(\vec{x}, \vec{z}) = \overline{h(\vec{z}, \vec{x})}$
	\end{enumerate}
	où $\overline{h(\vec{z}, \vec{x})}$ désigne le conjugué de $h(\vec{z}, \vec{x})$ dans $\mathbb{C}$.\\
	
	La condition \textit{3.} implique la \textbf{semi-linéarité à droite} de $h$:\\
	$\forall \vec{x}, \vec{y}, \vec{z} \in V\ et\ \forall \lambda \in \mathbb{C} :$
	\begin{enumerate}
		\item $h(\vec{x}, \vec{y} + \vec{z}) = h(\vec{x}, \vec{y}) + h(\vec{x}, \vec{z})$
		\item $h(\vec{x}, \lambda\vec{z}) = \overline{\lambda}h(\vec{x}, \vec{z})$
	\end{enumerate}
	Cette deuxième implication expliquera la présence du conjugué à droite dans l'écriture matricielle du produit hermitien. \\
	\textit{NB :} On général, on désigne les produits hermitiens par $< , >$ et non par $h$.
	
	\subsection*{Définition 2 (A connaître par keur-keur)}
	Soit $V = \mathbb{C}^n$, on appelle \textbf{produit hermitien usuel} l'application :
	$$< , > : \mathbb{C}^n  \times \mathbb{C}^n \rightarrow \mathbb{C} : (\vec{x}, \vec{y}) \rightarrow <\vec{x}, \vec{y}> = \sum_{i=1}^n x_i\overline{y_i}$$
	où $\vec{x}$ et $\vec{y}$ appartiennent à $\mathbb{C}$.
	
	\section{Matrice d'un produit hermitien dans une base donnée}
	C'est exactement la même chose que pour le produit scalaire si ce n'est qu'on utilise le produit hermitien.
	$$G^e = \left( <\vec{e_i}, \vec{e_j}>\right)$$
	
	On calcule ainsi le produit hermitien de la façon suivante : 
	$$<\vec{x}, \vec{y}> = \left(^t X_e\right)G^e\left(\overline{Y_e}\right)$$
	\textbf{Attention :} on conjugue les composantes de $\vec{y}$ !
	
	\subsection*{Théorème}
	On dira qu'une application dans une base $e$ est un produit hermitien sur $V$ ssi $H^e$ est un matrice hermétique ($H = ^t(\overline{A})$).
	
	\section{Matrice d'un produit hermitien et changement de base}
	Semblable aux produits scalaires, il ne faut juste pas oublier la conjuguée
	$$G^u =\ ^t(P^u_e) G^e (\overline{P^u_e})$$
	
	\section{Produits hermitiens définis positifs}
	Tatatitatata sera défini positif ssi :
	\begin{enumerate}
		\item $\forall \vec{x} \in V : <\vec{x}, \vec{x}> \in \mathbb{R}$
		\item $\forall \vec{x} \in V : \vec{x}, \vec{x}> \geq 0$
		\item $\forall \vec{x} \in V : \vec{x}, \vec{x}> = 0 \Leftrightarrow \vec{x} = \vec{0}$
	\end{enumerate}
	Par propriété, le produit hermitien usuel est défini positif.
	
	\section{Espaces hermitiens centrés}
	\textbf{Un espace hermitien centré} (ou \textbf{espace préhilbertien complexe}) est un espace vectoriel complexe muni d'un produit hermitien défini positif.
	
	
	\chapter{Orthogonalité et normes}
	\section{Espaces euclidiens et hermitiens centrés}
	Un espace \underline{euclidien centré} est un espace vectoriel réel muni d'un produit scalaire défini positif.\\
	Un espace \underline{hermitien centré} est un espace vectoriel complexe muni d'un produit hermitien défini positif.
	
	\section{Sous-espace orthogonaux}
	\subsection*{Définition 1}
	Si $V, < , >$ est un espace euclidien ou hermitien centré, alors deux vecteur $\vec{x}$ et $\vec{y}$ de $V$ sont \textbf{orthogonaux} \textit{pour} $< , >$ ssi $<\vec{x}, \vec{y}> = 0$.
	
	\subsection*{Définition 2}
	Si $S$ est un sous-ensemble non vide d'un espace euclidien ou hermitien centré $V, < , >$, alors 
	$$S^{\perp} =  \left\{\vec{y} \in V | <\vec{x}, \vec{y}> = 0\ \forall \vec{x} \in S \right\} = l'orthogonal\ de\ S\ pour\ <\ ,\ >$$
	
	\subsection*{Propriétés}
	Si $S$ est un sous-espace vectoriel d'un espace euclidien ou hermitien centré V, < , >, alors $S^{\perp}$ est aussi un sous-vectoriel de V, < , >. On appelle alors $S^{\perp}$ le \textbf{sous-espace orthogonal de} $S$.\\
	
	Si $S$ est un sous-espace vectoriel d'un espace euclidien ou hermitien centré V, <, >, alors 
	$$dim(S) + dim(S^\perp) = dim(V)$$.\\
	
	Si $S$ est un sous-ensemble non vide d'un espace euclidien ou hermitien centré V, <, > alors $S \cap S^\perp = \{\vec{0}\}\ si\ \vec{0} \in S\ ou\ \phi\ si\ \vec{0} \notin S$.\\
	
	Si $S = V$ alors $S^\perp = \{\vec{0}\}$. ; le vecteur nul est le seul vecteur d'un espace euclidien (ou hermitien) centré orthogonal à tous les vecteurs de l'espace.\\
	
	Si $V, <, >$ est un espace euclidien ou hermitien centré, alors tout sous-ensemble de vecteur \textbf{non nuls, orthogonaux deux à deux} forme une partie libre de $V$.
	
	\section{Projection orthogonale - Coefficient de Fourier}
	\subsection*{Définition 3}
	Si $V, < , >$ est un espace euclidien ou hermitien centré, et si $\vec{x}$ et $\vec{y}$ sont deux vecteurs \textit{non nuls} de V, le \textbf{coefficient de Fourrier de $\vec{x}$ par rapport à $\vec{y}$} vaut : 
	$$\lambda = \frac{\langle\vec{x}, \vec{y}\rangle}{\langle\vec{y}, \vec{y}\rangle}$$
	
	Si $V, \langle , \rangle$ est un espace euclidien ou hermitien centré, et si $\vec{x}$ et $\vec{y}$ sont deux vecteurs \textit{non nuls} de V,la \textbf{projection orthogonale de $\vec{x}$ su $\vec{y}$} est le \textsc{vecteur} :
	$$\vec{p} = \frac{\langle\vec{x}, \vec{y}\rangle}{\langle\vec{y}, \vec{y}\rangle}\vec{y}$$
	
	\subsection{Théorème important :}
	Si V, <, > est un espace euclidien ou hermitien centré, et si $W$ est un sous-espace de dimension $k$ de $V$, $W$ étant muni d'une base $(\vec{e_1'}, ..., \vec{e_k'})$ orthogonale pour $\langle , \rangle$ alors \textbf{la projection orthogonale d'un vecteur $\vec{x} \in V - W$ sur $W$} est égale au vecteur 
	$$\vec{p} = \sum_{i=1}^k \frac{\langle\vec{x}, \vec{e_i'}\rangle}{\langle \vec{e_i'}, \vec{e_i'}\rangle}\vec{e_i'}$$
	Ce vecteur est la somme des projections orthogonales du vecteur $\vec{x}$ sur chacun des vecteurs de la base $e'$.\\
	
	De façon plus générale, si $\vec{x} = \sum_{i=1}^n x_i\vec{e_i}$, alors : ( où $x_j = \frac{\langle\vec{x}, \vec{e_j} \rangle}{\langle\vec{e_j}, \vec{e_j}\rangle}$) :
	$$X_e = \begin{pmatrix} 
	x_1\\
	\vdots\\ 
	x_j\\
	\vdots\\
	x_n \end{pmatrix}$$
	\textbf{Voir conclusion page 211}.
	
	\subsection*{Définition 5}
	Soit $\mathbb{R}, V, +, \langle , \rangle$ un espace euclidien ou hermitien centré de dimension finie $n$. Une \textbf{base} $e = (\vec{e_1}, ... \vec{e_n})$ est \textbf{orthonormée pour le produit $\langle , \rangle$}
	$$\Leftrightarrow \langle\vec{e_i}, \vec{e_j}\rangle = \delta_{ij}$$
	
	\section{Expression d'un produit scalaire dans une base orthonormée}
	\subsection{Théorème (très très important)}
	Si $K , V, +, \langle , \rangle$ est un espace euclidien ou hermitien centré de dimension finie $n$, muni d'une base $e$ \textbf{orthonormée pour le produit $\langle , \rangle$} alors 
	$$\forall \vec{x}, \vec{y} \in V : \langle \vec{x}, \vec{y}\rangle = \sum_{i=1}^n x_i\overline{y_i}$$
	où $X_e$ est la matrice des composantes de $\vec{x}$ dans la base orthonormée $e$ et $Y_e$ est la matrice des composantes de $\vec{y}$ dans la base orthonormée $e$.\\
	
	En français : Certains produits scalaires peuvent être assez contraignants à calculer. Mais, si l'on construit une base orthonormée pour ce \textit{moche} produit, en travaillant dans cette nouvelle base le \textit{moche} produit \textbf{se ramène au produit scalaire ou hermitien usuel à partir des composantes de ces vecteurs dans la base orthonormée trouvée} ce qui simplifie les calculs ! Houra, me direz-vous.
	
	
	\subsection*{Application importante : Théorème}
	$$\forall \vec{x}, \vec{y} \in V : \pscal{\alpha(\vec{x})}{\vec{y}} = \pscal{\vec{x}}{\alpha(\vec{x})}$$
	Si ce théorème est vérifié, on dira que $\alpha$ est \textbf{auto-adjoint} (ou hermitien).
	
	\section{Algorithme d'orthogonalisation de Gram - Schmidt}
	Partie essentiellement pratique, \textit{cf. TP}.
	
	\section{Norme euclidienne ou hermitienne}
	Si $V, \langle , \rangle$ est un espace euclidien (ou hermitien) centré, on appelle \textbf{norme euclidienne} (... hermitienne) \textbf{associée à} $\langle , \rangle$ toute application : 
	$$||\ ||  : V \rightarrow \mathbb{R}^+ : \vec{x} \rightarrow ||\vec{x}|| = \sqrt{\pscal{\vec{x}}{\vec{x}}}$$
	
	\subsection{Propriété de la norme euclidienne (ou hermitienne)}
	C'est relativement simple, \textit{cf. page 221 - 227}.
	
	\section{Norme généralisée}
	Si $K, V, +$ est un espace vectoriel réel ou complexe, on appelle \textbf{borme de $V$} toute application $||\ || : V \rightarrow \mathbb{R}^+ : \vec{x} \rightarrow ||\vec{x}||$ telle que : 
	\begin{enumerate}
		\item $\forall \vec{x} \in V : ||\vec{x}|| \geq 0$ et $\left[||\vec{x}|| = 0 \Leftrightarrow \vec{x} = \vec{0}\right]$
		\item $\forall \vec{x} \in V$ et $\forall \lambda \in K : ||\lambda\vec{x}|| = |\lambda |\ ||\vec{x}||$
		\item $\forall \vec{x}, \vec{y} \in V : ||\vec{x} + \vec{y}|| \leq ||\vec{x}|| + ||\vec{y}||$\ \ (inégalité triangulaire)
	\end{enumerate}
	
	\chapter{Formes quadratiques}
	A partir d'ici, cette synthèse comprend seulement ce qui à une application aux TP, cela ne sert à rien de couvrir plus (Cf. la prof). En gros la matière théorique de l'examen de juin porte sur la matière des TP.
	
	\section{Définition - forme polaire}
	Si V, + est un espace vectoriel de dimension finie sur un corps K commutatif, tel que $0 \neq 2$, et si $f : V \times V \rightarrow K : (\vec{x}, \vec{y}) \rightarrow f(\vec{x}, \vec{y})$ est une forme bilinéaire \textbf{symétrique}, alors on appelle \textbf{forme quadratique sur V associée à} $f$ l'application $q : V \rightarrow K : \vec{x} \rightarrow q(\vec{x}) = f(\vec{x}, \vec{x})$.\\
	
	Ainsi, toute forme bilinéaire symétrique définit une forme quadratique et réciproquement, la donnée d'une forme quadratique permet de reconstituer univoquement la forme bilinéaire associée.
	
	\section{Formes quadratiques réelles}
	Il existe toujours une base $u$ de V telle que $Q^u = F^u$ est une matrice \textbf{diagonale}.
	
	\section{Réduction de Gauss}
	Voir séance de TP, mais il est très important d'appliquer l'algorithme \textsc{à la lettre} !\\
	Celle-ci permet de trouver une base dans laquelle la forme quadratique est diagonale.
	
	\section{Loi d'inertie - Signature d'une forme quadratique}
	Toute représentation en somme de carrés d'une forme quadratique réelle aura 
	\begin{enumerate}
		\item Un nombre constant $P_q$ de termes à coefficients strictement positifs
		\item Un nombre constant $N_q$ de termes à coefficients strictement négatif
	\end{enumerate}
	La \textbf{signature de la forme quadratique q} est le couple $(P_q, N_q)$.\\
	
	Le nombre de termes non nuls d'une représentation diagonale d'une forme quadratique réelle $q$ s'appelle le \textbf{rang de q}, qui est égal au rang de la matrice de $q$ dans n'importe quelle base de l'espace.
	
	
	\chapter{Valeurs propres et sous-espaces propre d'un opérateur linéaire}
	\section{Définitions}
	Si V, + est un espace vectoriel sur un corps K commutatif, et si $\alpha : V \rightarrow V$ est un opérateur linéaire de V, alors on appelle \textbf{vecteur propre de $\alpha$} tout vecteur $\vec{x}$ de V tel que $\exists \lambda \in K\ |\ \alpha(\vec{x}) = \lambda\vec{x}$.\\
	
	\textbf{Définition 2}\\
	Si V, + est un espace vectoriel sur un corps K commutatif, et si $\alpha : V \rightarrow V$ est un opérateur linéaire de V, alors un \textbf{scalaire} $\lambda \in K$ est une \textbf{valeur propre de $\alpha$} ssi il \textbf{existe} un vecteur $\vec{x} \in V$ \textbf{non-nul} tel que $\alpha(\vec{x}) = \lambda \vec{x}$.\\
	
	\textbf{Définition 3}\\
	Le \textbf{sous espace propre de $\alpha$ associé à la valeur propre $\lambda$} est l'ensemble de tous les vecteurs propres de $\alpha$ associé à la valeur propre $\lambda$
	$$W_\lambda = \left\lbrace \vec{x} \in V\ |\ \alpha(\vec{x}) = \lambda\vec{x}\right\rbrace$$\\
	
	\textbf{Définition 4}\\
	L'ensemble des valeurs propres de $\alpha$ s'appelle le \textbf{spectre de l'opérateur $\alpha$}.
	
	\section{Valeurs propres particulières}
	\begin{itemize}
		\item[$\lambda = 1$] L'ensemble des valeurs propre de $\alpha$ associée à $\lambda = 1$ est l'ensemble des \textit{points fixes de $\alpha$}.
		\item[$\lambda = 0$] L'ensemble des valeurs propre de $\alpha$ associée à $\lambda = 0$ est \textit{le noyau $ker(\alpha)$ de $\alpha$}.
		\item[$\lambda = 0$] L'application $\alpha$ n'est pas injective (il faut en effet qu'il existe un vecteur non nul, ce qui n'est pas le cas ici)
	\end{itemize}
	
	
	%%%%%%%%%%%%%%%%%
	% Bibliographie %
	%%%%%%%%%%%%%%%%%
	%\newpage
	%\chapter{Bibliographie}
	%\nocite{*}
	%\printbibliography[heading=none]
	
	%%%%%%%%%%%
	% Annexes %
	%%%%%%%%%%%
	\appendix
	%\input{annexes/annexe1.tex}
\end{document}
