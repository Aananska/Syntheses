
\chapter{Centrifugal pumps}
\section{Generalities}
\subsection{Description - Type of turbopump}
\minifig{ch2/1}{ch2/2}{0.3}{0.4}{0.2}{0.3}
The task of the turbopump is to transfer energy to a liquid. Above we can see a centrifugal and an axial turbopump. Between these two extremes, we can have a variety of of types depending on the requirements. Each turbopump is composed of one or several wheels that can be mounted in parallel (increase mass flow rate) or in series (higher energy transfer), see \autoref{ch1/2}. 

\subsubsection{Installation of a turbopump}
\minifig{ch2/3}{ch2/4}{0.3}{0.4}{0.2}{0.3}
The general scheme is shown here, observe that the flow enters at the middle in the rotating blade and is projected into the volute. This last has a growing section from the beginning to the end as the mass flow increases. The turbopump is commonly used to transfer liquid from a downstream reservoir to an upstream reservoir situated higher. We have to be careful to avoid cavitation (absence of liquid in the inlet pipe) and thus we have a control valve at the suction section. 