
\chapter{Elements of PDE's}

Convergence was an interesting issue but the type of the equation we have to solve is also important. This is why we study the elements of the theory of PDE’s

\section{Quasi-linear equations – Conservative form}
A quasi-linear equation is an equation that is linear in the highest derivatives. We start with a first order equation. 

\subsubsection{General form of a first order quasi-linear equation in two variables}
For example for a two space variables, it means that we look for a function of $u(x,y)$ involving only first derivatives. And typically we have a linear combination of $x$ and $y$ derivatives and the coefficients may depend on $u$ too:

\begin{equation}
P(x,y,u) \frac{\D u}{\D x} + Q(x,y,u) \frac{\D u}{\D y} = R(x,y,u)
\end{equation}

This doesn't mean that the equation is linear, for example $\frac{\D u}{\D x} + u \frac{\D u}{\D y} = S(x,y,u)$ non linear Burger's equation. 

\subsubsection{General form of a second order quasi-linear equation in two variables}

The second order equation in 2 space variables: 

\begin{equation}
\begin{aligned}
P\left(x,y,u, \frac{\D u}{\D x}, \frac{\D u}{\D y}\right) \frac{\D u^2}{\D x^2} &+ 2 S\left(x,y,u, \frac{\D u}{\D x}, \frac{\D u}{\D y}\right) \frac{\D u^2}{\D x \D y}\\
&+ T\left(x,y,u, \frac{\D u}{\D x}, \frac{\D u}{\D y}\right) \frac{\D u^2}{\D y^2} = H\left(x,y,u, \frac{\D u}{\D x}, \frac{\D u}{\D y}\right)
\end{aligned}
\end{equation}

In most applications in fluid mechanics, the factors multiplying the higher order derivatives do not depend explicitly on the independent variables $x,y$. But they will depend implicitly since $u$ is a function of $x,y$. Let's give an example of this, 2D potential equation for compressible flows: 

\begin{equation}
(a^2 - u^2)\frac{\D^2\varphi}{\D x^2} - 2uv \frac{\D ^2 \varphi}{\D x \D y} + (a^2 - v^2) \frac{\D ^2 \varphi}{\D y^2} = 0
\end{equation}

Quasi-linear equations can appears under the conservative/divergence form would be in the form 

\begin{equation}
\frac{\D g_x}{\D x} + \frac{\D g_y}{\D y} (=\nabla .\vec{g}) = S(x,y,u)
\end{equation}

where $g_x, g_y = \vec{g}(x,y,u) = \vec{g}\left(x,y,u,\frac{\D u}{\D x}, \frac{\D u}{\D y}\right)$ for respectively a first order and a second order equation. Notice that it is possible to recover from here the quasi-linear form. Let's call $\tilde{g}_x (x,y) = g_x(x,y,u(x,y))$ first order equation then we have that by chain rule that: 

\begin{equation}
\frac{\D \tilde{g}_x}{\D x} = \frac{\D g_x}{\D x}  +\frac{\D g_x }{\D u}\frac{\D u}{\D x} \qquad \frac{\D \tilde{g}_y}{\D y} = \frac{\D g_y}{\D y}  +\frac{\D g_y }{\D u}\frac{\D u}{\D y}
\end{equation}

Then the sum of the two gives: 

\begin{equation}
\begin{aligned}
\frac{\D \tilde{g}_x}{\D x} + \frac{\D \tilde{g}_y}{\D y} = \frac{\D g_x}{\D x} + \frac{\D g_y}{\D y}  +\frac{\D g_x }{\D u}\frac{\D u}{\D x} +\frac{\D g_y }{\D u}\frac{\D u}{\D y}  \\
\frac{\D \tilde{g} _x}{\D u} \frac{\D u}{\D x} + \frac{\D \tilde{g}_y}{\D u} \frac{\D u}{\D y} = S - \frac{\D g_x}{\D x} - \frac{\D g_y}{\D y}
\end{aligned}
\end{equation}

where we refined our coefficients $P$ and $Q$. For a first order equation in 2 space variables the converse is true as well. Indeed, defining $P= \frac{\D \hat{P}}{\D u}$ and $Q= \frac{\D \hat{Q}}{\D u}$, the chain rule is: 

\begin{equation}
\frac{\D \tilde{P}}{\D x} = \frac{\D \hat{P}}{\D x}  +\frac{\D \hat{P}}{\D u}\frac{\D u}{\D x}= \frac{\D \hat{P}}{\D x} + P\frac{\D u}{\D x} \qquad \Rightarrow P\frac{\D u}{\D x} = \frac{\D \tilde{P}}{\D x} - \frac{\D \tilde{P}}{\D x}
\end{equation}

Replacing these in the general first order equation form, we find back the conservative form: 

\begin{equation}
\frac{\D \tilde{P}}{\D x} + \frac{\D \tilde{Q}}{\D y} = S + \frac{\D \hat{P}}{\D x} + \frac{\D \hat{Q}}{\D y}
\end{equation}

For a second order equation, this is not always possible. 

\section{Characteristic surfaces and wave-like solutions}
\subsection{First order scalar equation in m independent variables}

It means that we have a linear combination such as: 

\begin{equation}
a_i \frac{\D u}{\D x_i} = 0 
\end{equation}

For simplicity we take the source term $=0$. We suppose to solve an initial value problem (Cauchy problem). We imagine that the solution is known on some hyper-surface $S^*$ (in 2D a curve) of equation $F(x_i) = 0$. Does this problem have one and only one solution? \\

The value of $u$ on the surface is called the \textbf{trace} of the solution on the hyper surface and it is specified. Imagine that we can construct a function $\varphi (x,y)$ such that it is equal to $u$ on the surface. Typically if we think in 2D, we specify $u$ on a curve and we elongate it arbitrarily. Because the function $\varphi = u$ on the surface, then the tangential derivative of $\varphi$ and $u$ are the same. If we construct the function $\varphi - \lambda F$ is also equal to $u$ on the surface whatever the value of $\lambda$. In other words, I know u on the curve, I construct a function $\varphi$ and then I say that all the function $\varphi - \lambda F$ are the same as $u$, so we have an infinite number of solution on the surface. But there exists one $\lambda$ for which the normal derivative will be the same as the normal derivative of $u$: 

\begin{equation}
\nabla \varphi - \lambda \nabla F = \nabla u \qquad \Leftrightarrow \frac{\D \varphi}{\D x_i } - \lambda \frac{\D F}{\D x_i } = \frac{\D u}{\D x_i} \quad i = 1,\dots, m
\end{equation} 

The unknowns in this equation are the partial derivatives of $u$ but are given by $a_i \frac{\D u}{\D x_i} = 0$. We have thus a system of $m+1$ equations and $m+1$ unknowns. The system can be put under matrix form as:

\begin{equation}
\left(
\begin{array}{ccccc}
1 & \dots & & & \frac{\D F}{\D x_1} \\
 & 1 &  & & \frac{\D F}{\D x_2}\\
 \vdots & & & & \vdots \\
 a_1 & a_2 & \dots & a_m & 0
\end{array}
\right)
\left(
\begin{array}{c}
\frac{\D u}{\D x_1}\\
\vdots\\
\frac{\D u}{\D x_m}\\
\lambda
\end{array}
\right)
=
\left(
\begin{array}{c}
\frac{\D \varphi}{\D x_1}\\
\vdots\\
\frac{\D \varphi}{\D x_m}\\
0
\end{array}
\right)
\end{equation}

The system has only one and only one solution unless if the determinant is equal to 0, unless the surface S is such that $a_i \frac{\D F}{\D x_i} = 0 = \vec{a}.\nabla F$. In fact $\nabla F$ is parallel to the normal to the surface because all the tangential derivatives are 0 and the only component that cannot be 0 is the normal derivative, so $\nabla F \propto \vec{n}$. And if we say $\vec{a}.\vec{n} = 0$,
this means that the surface is tangent to $\vec{a}$. These are called \textbf{characteristic lines} of a \textbf{characteristic surface}. The response to the question is thus that yes the solution is unique unless if the surface is a characteristic surface. In 2D this is a characteristic curve. \\

The original equation admits solutions of the form: $u = \hat{u}\exp (IF(x_i))$ where $F(x_i) = 0$ are equations of characteristic surfaces. Because if we compute $\frac{\D u}{\D x_i} = I u \exp (IF(x _i))\frac{\D F}{\D x_i} = Iu\frac{\D F}{\D x_i}$ and thus 

\begin{equation}
a_i \frac{\D u }{\D x_i} = IU \underbrace{a_i \frac{\D F}{\D x_i}}_{=0}
\end{equation}

Lines of constant $F$ are wave fronts of wave-like solutions since we have $\exp (I F(x_i))$ similar to $\exp (kx - \omega t)$. A special case is when $a_i = cst \rightarrow \frac{\D F}{\D x_i = n_i}$ where we have planar wave that propagates without dilatation or damping.  

\subsection{Second order equations in one unknown in two dimension}
Consider the second order equation in two variables. That is the equation in the form:

\begin{equation}
R\frac{\D ^2 u}{\D x^2} + 2S \frac{\D ^2 u}{\D x\D y} + T \frac{\D ^2 u}{\D y^2} = 0
\end{equation}

In this case we have to provide $u$ and $\nabla u$ but if we know the surface we can compute $\nabla u$. Let's call $p = \frac{\D u}{\D x}$ and $q = \frac{\D u}{\D y}$ on the curve. Let's call $\varphi$ a function equal to $p$ on C. It results that $\varphi = p + \lambda F$ on C and therefore there exists a value of $\lambda$ such that $\nabla \varphi = \nabla p + \lambda \nabla F$. So we have also for the second variable: 

\begin{equation}
\psi = q + \mu F \qquad \Rightarrow \nabla \psi = \nabla q + \mu \nabla F
\end{equation}

The unknowns are $\lambda , \mu$, the components of the gradients $p$ and $q$. But we know that 

\begin{equation}
\nabla p = \frac{\D ^2 u}{\D x^2} \vec{e}_x + \frac{\D ^2 u}{\D x \D y} \vec{e}_y\qquad \nabla q = \frac{\D ^2 u}{\D y^2} \vec{e}_y + \frac{\D ^2 u}{\D x \D y} \vec{e}_x
 \end{equation}
 
Again we have a system of equation

\begin{equation}
\left( 
\begin{array}{ccccc}
1 & 0 & 0 & \frac{\D F}{\D x} & 0\\
0 & 1 & 0 & \frac{\D F}{\D y} & 0 \\
0 & 1 & 0 & 0 & \frac{\D F}{\D x} \\
0 & 0 & 1 & 0 & \frac{\D F}{\D y} \\
R & 2S & T & 0 & 0
\end{array}
\right)
\left( 
\begin{array}{c}
\frac{\D p}{\D x} \\
\frac{\D p}{\D y} = \frac{\D q}{\D x}\\
\frac{\D q }{\D y}\\
\lambda\\
\mu 
\end{array}
\right)
=
\left( 
\begin{array}{c}
\frac{\D \varphi}{\D x} \\
\frac{\D \varphi}{\D y} \\
\frac{\D \psi}{\D x}\\
0\\
0
\end{array}
\right)
\end{equation}

If we try to compute the determinant we will get 

\begin{equation}
\det = R \frac{\D F}{\D x} \left( - \frac{\D F}{\D x} \right) - 2 S \frac{\D F}{\D x}\frac{\D F}{\D y}  +T \left( -\frac{\D F}{\D y}\right)\frac{\D F}{\D y} = - \left(\frac{\D F}{\D y}\right) ^2 \left[ Rz^2 + 2 Sz 
+ T \right]
\end{equation}

where $Z = \frac{\frac{\D F}{\D x}}{\frac{\D F}{\D y}} = \frac{n_x}{n_y}$. The line $F(x,y) = 0$ is a characteristic line when $Rz^2 + 2Sz + T = 0$. If $S^2 - RT > 0$ so we have 2 roots and thus 2 characteristic directions at each point. In the case $S^2 - RT < 0$ we have no real root no characteristic line for this equation. And if $S^2 - RT = 0$ we have 2 identical roots so 1 characteristic direction. If now we have a quadratic term like 

\begin{equation}
R x^2 + 2S xy + Tx^2 = 0
\end{equation}

The first case would give a hyperbole, the second an elliptic equation and the last we have a parabolic equation. 

\subsubsection{Applicaition: potential flow equation}

\begin{equation}
\begin{array}{c}
R = a^2 - u^2 \qquad S = -uv \qquad T = a^2 - v^2 \\ 
\Rightarrow S^2 - RT = u^2v^2 - (a^2 -u^2)(a^2 - v^2) = a^2 [u^2 + v^2 - a^2] = a^4 \left[M^2 - 1 \right]
\end{array}
\end{equation}

We can see that when $M>1$ hyperbolic, $M = 1$ parabolic, $M<1$ elliptic. 

\subsection{System of first order equations in two dimensions}
We can write it in the form of a system of n equations and n unknowns:

\begin{equation}
A_x \frac{\D U}{\D x}  +A_y \frac{\D U}{\D y} = 0
\end{equation}

It is now $U$ that is provided on the curve C of equation $F(x,y) = 0$. We call $V$ a vector function which is identical to $U$ on C. We are going to have a vector of Lagrange multipliers $\Theta$: 

\begin{equation}
V = U + \Theta F \mbox{ on C} \qquad \Rightarrow \frac{\D V}{\D x} = \frac{\D U}{\D x} + \Theta \frac{\D F}{\D x} \qquad \frac{\D V}{\D y} = \frac{\D U}{\D y} + \Theta \frac{\D F}{\D y}
\end{equation}

We will again have the matrices: 

\begin{equation}
\left( 
\begin{array}{ccc}
I & 0 & \frac{\D F}{\D x}I\\
0 & I & \frac{\D F}{\D y} I\\
A_x & A_y & 0
\end{array}
\right)
\left(
\begin{array}{c}
\frac{\D U}{\D x}\\
\frac{\D U}{\D y}\\
\Theta
\end{array}
\right)
=
\left(
\begin{array}{c}
\frac{\D V}{\D x}\\
\frac{\D V}{\D y}\\
0
\end{array}
\right)
\end{equation}

Again we have to compute the determinant. The characteristic directions are $|A_x n_x + A_y n_y| = 0$ and if we make the evidences: 

\begin{equation}
n_x |A_x + \lambda A_y| = 0 \qquad n_y \left|-A_x \frac{-n_x}{n_y} + A_y\right| = 0
\end{equation} 

where $\frac{-n_x}{n_y}$ is the slope of the tangent to the characteristic lines. We end up with an eigenvalue problem

\begin{equation}
|A_y - \lambda A_x| = 0
\end{equation}

if $\det (A_x) \neq 0$ so $|A_y - \lambda A_x| = |A_x||A_x^{-1}A_y - \lambda I| = 0$. We have n real roots, n lin indep real eigenvectors (hyperbolic) or n real roots, m<n lin indep real eigenvectors (parabolic). 

\begin{equation}
\begin{aligned}
R\frac{\D ^2 u}{\D x^2} + 2S \frac{\D ^2 u}{\D x \D y} + T\frac{\D ^2 u}{\D y^2} = 0 \qquad &\Leftrightarrow R\frac{D p}{\D x}  + S \left(  \frac{\D p }{\D y	} + \frac{\D q}{\D x}\right) + T\frac{\D q}{\D y} = 0\\
\left[
\begin{array}{cc}
R & S\\
0 & 1
\end{array}
\right]
\frac{\D }{\D x}\left( 
\begin{array}{c}
p\\
q
\end{array}
\right)
&+ \left( 
\begin{array}{cc}
S & T\\
1 & 0
\end{array}
\right)\frac{\D q}{\D x} - \frac{\D p }{\D y} = 0
\end{aligned}
\end{equation}

we also have the case elliptic problem with n complex roots, other possibilities = hybrid. \\

\subsection{Systems of n equations in m independant variables}
In that case the generalization is: 

\begin{equation}
A_i \frac{\D U}{\D x_i} = 0
\end{equation}

where $A$ is n by n matrix and $U$ a n by 1 vector. The equations for the characteristic surfaces is now: 

\begin{equation}
|1_i n_i| = 0
\end{equation}

Here because the n are components of the unit normal, but in x, y they are associated to one vector. If now we devide by $n_1$ we would have $|A_i n_i/n_1| = 0$ meaning that we would have $m-1$ variables. To illustrate suppose that we have: 

\begin{equation}
|A_x n_x + A_yn_y + A_zn_z | = 0
\end{equation}

We can write this in the following way: 

\begin{equation}
\left|\frac{A_x n_x + A_y n_y}{n_z } + A_z\right| = 0 \qquad \Rightarrow \left|\frac{A_x n_x + A_y n_y}{\sqrt{n_x^2 + n_y^2}}- \frac{\sqrt{n_x^2 + n_y^2}}{n_z} + A_z\right| = 0
\end{equation}

where in fact the first term is the direction of $n$ and the second term plays the role of the previous $\lambda$. If it is in 4D, we would have 2 angular directions defining the $n$ direction. $m-2$ variables can be chosen arbitrarily and the last is given by the eigenvalue problem. The system is hyperbolic wrt to $A_z$ if the eigenvalues all real and we have a complete set of eigenvectors for all directions whatever the direction. If in contrast if they are all complex whatever the direction, it is indetermined. The Euler equations will be: 

\begin{equation}
\frac{\D U}{\D t } + \frac{\D F}{\D x} = 0 \qquad 
U = \left(
\begin{array}{c}
\rho \\
\rho u\\
\rho E
\end{array}
\right)
\quad F = \left(
\begin{array}{c}
\rho u\\
p + \rho u 2\\
\rho uH
\end{array}
\right)
\end{equation}

in 1D unsteady z is an independent variable. We also know that $p = \rho RT = \rho (c_p - c_v) T$ and $z = c_v T = \rho (\gamma -1)c_v T$ and $p =\rho (\gamma -1)e, E = e+\frac{u^2}{2}, H = E + \frac{p}{\rho}$, so that we have: 

\begin{equation}
\frac{\D F}{\D U} = \left( 
\begin{array}{ccc}
0 & 1 & 0\\
\frac{\D p}{\D e} - u^2 & 2u + \frac{\D p}{\D \rho u} & \frac{\D p}{\D \rho E}\\
\times & \times & \times
\end{array}
\right)
\end{equation}

And we see that the eigenvalues are real: $u+a, u , u-a$. In 2D the $U$ and $F$ vectors are $4\time 1$ and thus the matrix will be $4\times 4$ : 

\begin{equation}
\begin{aligned}
&2D: \quad \frac{\D U}{\D t} + \frac{\D F_x}{\D x}+ \frac{\D F_y}{\D y}  = 0\\
&3D: \quad \frac{\D U}{\D t} + \frac{\D F_x}{\D x}+ \frac{\D F_y}{\D y} + \frac{\D F_z}{\D z}  = 0
\end{aligned}
\end{equation}

They are both unsteady with 3 and 4 independant variables. We see that these are hyperbolic wrt the time variable! \\

\theor{
\textbf{Conclusion}\\

If the system has real eigenvalues and a complete set of real eigenvalues for all values of the arbitrary parameters (m-2) $\rightarrow$ the system is hyperbolic wrt the variable of interest $\rightarrow$ this variable plays a special role, it is the evolution or time-like variable. \\

If the system has only complex eigenvalues the equation is elliptic. In that case, it is generally elliptic wrt all variables. 
}

\subsection{Notion of well posed problem}
Problem that has one and only one solution, depending continuously on the prescribed initial/boundary data. And they give the example of Laplace's equation: 

\begin{equation}
\frac{\D ^2 u}{\D x^2} + \frac{\D ^2 u}{\D y^2} = 0
\end{equation}

and we compute the solution on the domain $\Omega$ which is the right half plane where $x>0$. The boundary of the domain is $u(0,y) = 0$ and $\frac{\D u}{\D x}(0,y) = h(y)$. Since it is elliptic equation it has not real characteristic curve so we know that the curves are not characteristic and so that we have one and only one solution. If we take $h(y) = 0$ we have $u=0$, if we take $h(y) = \frac{\sin ny}{n}$ then $u = \frac{1}{n^2}\sinh nx \sin ny$. As $n\rightarrow \infty, h \rightarrow 0$ if we perturb the boundary data infinitesimally the solution will vary a lot. When we are never sure that the solution depends continuously on the boundary condition and when we have uncertainties on the boundary conditions we can never be sure of the solution. This is why if we have 2 boundary we have to impose one condition on each boundary. 

\section{Properties of hyperbolic equations}
\subsection{Nature of the solution - Riemann invariants}
For simplicity, consider a system of n equations in 2 independent variables $|A_x n_x + A_y n_y| = 0 \Leftrightarrow |A_y - \lambda A_x| = 0$ where $\lambda = -n_x/n_y$. If $A_x$ is positive definite regular matrix: 

\begin{equation}
|A_x (A_x^{-1}A_y - \lambda I)| = 0 \qquad \Rightarrow |A_x||\underbrace{A_x^{-1}A_y}_{A}-\lambda I| = 0
\end{equation}

The $\lambda _s$ are the eigenvalues of $A$. Calling $|A-\lambda I| \qquad Av = \lambda v$ where $v$ are the right eigenvectors and $l$ the line eigenvectors: $lA = \lambda l \Leftrightarrow (lA)^t = \lambda l^t$. We have the following algebraic identities: 

\begin{equation}
Av_1 = \lambda _1v_1 \quad Av_2 = \lambda _2v_2 \dots \qquad \Rightarrow A R = R \Lambda \Leftarrow R^{-1}AR = \Lambda \Leftarrow R^{-1} A = \Lambda R^{-1}
\end{equation}

The same can be done for the left eigenvalues and we can find the relation $L = R^{-1}$. Let's imagine that the initial problem to solve was (as in aerodynamics course - Riemann): 

\begin{equation}
A_x \frac{\D U}{\D x} + A_y \frac{\D U}{\D y} = S\qquad \Leftrightarrow  \frac{\D U}{\D x} + A \frac{\D U}{\D y} = \underbrace{A_x^{-1} S}_G \Leftrightarrow L\frac{\D U}{\D x} + \Lambda L \frac{\D U}{\D y} = LG
\end{equation}

If now we consider the first eigenvalue: 

\begin{equation}
\begin{aligned}
l_{11} \frac{\D u_1}{\D x} + l_{12} \frac{\D u_2}{\D x} + \dots + l_{1n} \frac{\D u_n}{\D x} + \lambda \left[ l_{11} \frac{\D u_1}{\D y} + \dots \right] = (LG)_1 \\
l_{11} \frac{\D u_1}{\D x} + \lambda l_{11} \frac{\D u_1}{\D y} = l_{11} \left( \frac{\D u_1}{\D x} + \lambda \frac{\D u_1}{\D y}\right) = l_{11} \left( \frac{\D u_1}{\D x} + \tan \theta \frac{\D u_1}{\D y}\right) \\
= \frac{l_{11}}{\cos \theta} \left(\cos \theta \frac{\D u_1}{\D x} + \sin \theta\frac{\D u_1}{\D y}\right)
\end{aligned}
\end{equation}

where $\frac{1}{\cos \theta} = \sqrt{1+\lambda ^2}, \vec{e}_\theta .\nabla u = \frac{du_1}{dS_1}$. This can be done for every indexes and we get: 

\begin{equation}
\frac{1}{\cos \theta} \left( l_{11} \frac{du_1}{dS_1} + \dots + l_{1n}\frac{du_n}{dS_n} \right) = h_1
\end{equation}

For a system of n equations, the system transforms into a system of n ordinary differential equations: 

\begin{equation}
l_{ij} \frac{d u_j}{dS_{i}} = h_i \cos \theta _i = \frac{h_i}{\sqrt{1+\lambda ^2_i}}
\end{equation} 

Can we simplify further? If $l_{ij}$ do not depend explicitly on the independent variables $x,y$ (they depend only on $u_j$), there may exist a function $f(u_j)$ such that \begin{equation}
fl_{ij} \frac{du_j}{dS_i} = \frac{dR_i}{dS_i}
\end{equation}

with the conditions: 

\begin{equation}
fl_{ij} = \frac{\D R_i}{\D u_j} \Rightarrow \frac{\D fl_{ij}}{\D u_k} = \frac{\D fl_{ij}}{\D u_j} \qquad fl_{ik} = \frac{\D R_i}{\D u_k}
\end{equation}

$f$ is called an integrating factor. If $n\ll 2\rightarrow f$ always exists, if $l_{ij}$ are constant $\forall n, R_i = l_{ij} u_j$. We finally have 

\begin{equation}
\frac{d R_i}{ds_i} = \frac{h_i}{\sqrt{1+\lambda ^2_i}} \qquad \Rightarrow R_i = cst
\end{equation}

$R_i$ are the Riemann invariants. The summary is given on the figure p55. Second figure, let's try to think about the properties of the solution. Imagine that the solution here for 2 unknowns, so 2 characteristics. Let's think that the solution is prescribed on the curve $\Gamma$, the solution at point P has to depend on all the values in the dark region since we have an integration. The rest of the region is the region of silence. \\

Consider 1D inviscid flow in a tube. Continuity, x-momentum and energy equations tells that: 

\begin{equation}
\frac{\D \rho}{\D t} + \frac{\D \rho u}{\D x} = 0 \qquad \rho \frac{\D u}{\D t} + \rho u \frac{\D u }{\D x} = - \frac{\D p}{\D x}= -a^2\frac{\D \rho}{\D x} \qquad \dot{s} = 0
\end{equation}

If initial data are homentropic (uniform initial entropy): $s = cst$. We can rewrite: 

\begin{equation}
\begin{aligned}
&\frac{\D \rho}{\D t} + \frac{\D \rho u}{\D x} = 0 \qquad \frac{\D u}{\D t} + u \frac{\D u }{\D x} + \frac{a^2}{\rho} \frac{\D \rho}{\D x} = 0\\
&\Rightarrow \frac{\D }{\D t} \left( 
\begin{array}{c}
\rho \\
u
\end{array}
\right) + \left( 
\begin{array}{cc}
u & \rho\\
\frac{a^2}{\rho} & u
\end{array}
\right) 
\frac{\D}{\D x} \left( 
\begin{array}{c}
\rho \\
u
\end{array}
\right) 
=
\left( 
\begin{array}{c}
0 \\
0
\end{array}
\right) 
\end{aligned}
\end{equation}

After computing the eigenvectors we found that they are: $\lambda = u\pm a$. If we make the drawing, we have the graph of x in function of t and since the length is limited on top at L due to energy, we have to specify the upper and lower boundary to be able to compute in all the domain. We have to supply as many information as the number of characteristic curves entering the domain. \\

DO THE NON-LINEAR HE READS THE SYLLABUS he stoped at the graphs. 

We see on the right that we can find a continuous function to define the transition region. But in the right hand side we could have 2 solutions, not unically defined solution so we need to look for a weak solution. This is done by weightening the function in a certain way: 

\begin{equation}
\int _0 ^\infty dt \int _{-\infty}^\infty \nu \left[\frac{\D u}{\D t} + u \frac{\D u }{\D x}\right] dx = 0 
\end{equation}

Then we can integrate by part and find that: 

\begin{equation}
\int \int \nu \left[ \frac{\D u}{\D t} +\frac{\D u^2/2}{\D x}\right] dx\, dt = \int \int \left[ \frac{\D uv}{\D t} + \frac{\D}{\D x} \frac{u^2}{2} \nu  \right]dx\, dt - \int \int \left[ u\frac{\D \nu}{\D t} + \frac{\D \nu}{\D x} \frac{u^2}{2} \nu  \right]dx\, dt  =0 =\int _0^\infty  \int _{-\infty}^\infty \left[u \frac{\D \nu}{\D t} + \frac{u^2}{2}\frac{\D \nu }{\D x}\right] dx\, dt = -\int _{-\infty}^\infty u(0,x)\nu (0,x) \, dx +0
\end{equation}

Integral: 

\begin{equation}
\frac{d}{dt} \int _{x_e}^{x_r} u \, dx + \left[ \frac{u^2}{2}\right]_{x_e}^{x_r} = 0 
\end{equation}

Do all the text alone mohofokor. 

\section{Properties of elliptical equations}
\subsection{Nature of the solution}
We want to discuss the nature of the solutions and we consider the Laplace equation as example: 

\begin{equation}
\Delta u = \nabla ^2 u = \frac{\D^2 u}{\D x^2} + \frac{\D^2 u}{\D y^2}
\end{equation}

We consider the similar problem as previously $u(0,y) = g_1(y)$ and $\frac{\D u}{\D x} (0,y) = g_2 (y)$. We will use a Fourrier transform: 

\begin{equation}
u(x,y) = \frac{1}{2\pi}\int _{-\infty} ^\infty \hat{u} (x,\omega) e^{i\omega y} \, d\omega \qquad \frac{\D u }{\D y} = \frac{1}{2\pi} \int _{\infty} ^\infty   i \omega \hat{u} (x,\omega )e^{i\omega y} \, dy \qquad \frac{\D^2 u}{\D y^2} = \frac{1}{2\pi} \int _{\infty} ^\infty   - \omega ^2\hat{u} (x,\omega )e^{i\omega y} \, dy
\end{equation}

So that the equation becomes: 

\begin{equation}
\frac{\D ^2 \hat{u}}{\D x^2} - \omega ^2 \hat{u} = 0 \qquad \hat{u} = A(\omega) e^{\omega x} + B(\omega) e^{-\omega x}
\end{equation}

and thus 

\begin{equation}
u(x,y) = \frac{1}{2\pi } \int _{-\infty} ^\infty \left[A(\omega ) e^{\omega x} + B(\omega ) e^{-\omega x} \right] e^{i\omega y} \, d\omega
\end{equation}

where 

\begin{equation}
A(\omega) +B(\omega ) = \int _{-\infty} ^{\infty} g_1(y) e^{-i\omega y} \, dy \qquad \Rightarrow u(0, y) = g_1(y) = \frac{1}{2\pi} \int _{-\infty} ^{\infty} [A(\omega) + B(\omega)] e^{i\omega y} \, d\omega \qquad \frac{\D u}{\D x} (0,y) = g_2(y) = \frac{1}{2\pi} \int _{-\infty }^{\infty} \omega [A(\omega) - B(\omega) ]e^{i\omega y} \, dy
\end{equation}

to have finally 

\begin{equation}
\frac{\D u}{\D x} = \frac{1}{2} \int _{-\infty} ^\infty \omega [A(\omega) e^{\omega x}-B(\omega) e^{\omega x}] e^{i\omega y}\, dy
\end{equation}

If we introduce a perturbation into the domain it will be felt in the whole domain, the region of dependance is the entire domain. This is a completely different behavior than the previous case. We have some examples: if we inject a heating resistor in a pool, we will have a temperature increase in the whole pool, but since it increases exponentially from the source we don't feel it far away. 

\subsection{Well posed problem for an elliptic system}
For this we have to impose the second condition so that the expansion due to the exponent is canceled. 

\subsubsection{Existence of the solution}
If we know the solution on the boundary we can compute everywhere. 

\subsubsection{Solution uniqueness}
\subsubsection{Continuity in the sense of Hadamard}

\section{Parabolic equations}

\section{•}
Consider the following equation, 2D small disturbance potential equation: 

\begin{equation}
(1-M^2_\infty) \frac{\D^2 \phi}{\D x^2} + \frac{\D^2\phi}{\D y^2} = 0 \qquad S^2 - RT = M^2_\infty -1 
\end{equation}

$>0$ (hyperbolic) $M_\infty^2>1$, $<0$ (elliptic) $M_\infty^2<1$. Consider a mesh and we discretize by central differences on a coarse 4x4 mesh. Normally we have 2 indexes but here we consider the points by full number, and we don't need the corners. If I consider a generic point O, I will neighbour the points by N W E S and the expression will be: 

\begin{equation}
(1 - M_\infty^2) \frac{\phi _W + \phi _E - 2\phi _O}{\Delta x^2} + \frac{\phi _N + \phi _S - 2\phi _O}{\Delta y^2} = 0 \qquad \Delta x = \Delta y = h \qquad \rightarrow (1-M_\infty ^2) [\phi _W + \phi _E - 2\phi _O][\phi _N + \phi _S - 2\phi _O] = 0
\end{equation}

If the equation is elliptic we just have to consider the sides and we know the $\phi$ on the sides. We can build a matrix, since we know the values on the sides we can just put 1 on diagonals for 3 first and then we go into internal points and we make the equation: 

put the equation in figure. \\

For supersonic case, we know $\frac{\D \phi }{\D x}$ on point 1: 

\begin{equation}
\phi _2 + \phi _c - \phi _4 - \phi _a = 0 \qquad \left.\frac{\D \phi}{\D x} \right| _1 = \frac{\phi _4 - \phi _a}{2h} = u_1 \mbox{ (specified)}
\end{equation}

He showed some slides 


