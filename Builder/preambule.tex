%%%%%%%%%%%%%%%%
%%% Packages %%%
%%%%%%%%%%%%%%%%

%%% Compatibilité %%%
\begingroup\expandafter\expandafter\expandafter\endgroup
\expandafter\ifx\csname IncludeInRelease\endcsname\relax
\usepackage{fixltx2e}
\fi 					% Si version LaTeX < 2015, inclut un fix.

%%% Général %%%
\usepackage[utf8]{inputenc}
\usepackage{babel}
\usepackage{lmodern}
\usepackage[T1]{fontenc}
\addto\extrasfrench{\sisetup{locale = FR,detect-all}} % Switch siunitx en fonction de la langue babel :)
\addto\extrasbritish{\sisetup{locale = UK,detect-all}}
\usepackage{courier}
\usepackage{graphicx}
%\usepackage{cancel}

%%% Tableau %%%
%\usepackage{tabularx} %Permet d'auto dimensionner les tableaux



%%% Bibliographie %%%
%\usepackage[style=alphabetic,backend=biber]{biblatex}
\usepackage[autostyle]{csquotes}
%\DeclareNameAlias{sortname}{last-first}
%\DeclareFieldFormat{url}{\space\url{#1}}
%\DeclareNameAlias{labelname}{last-first}
%\addbibresource{sample.bib}


%%% Graphiques %%%
%\usepackage{tikz}
%\usepackage{pgfplots}
%\usepackage{circuitikz}

%%% Mise en page %%%
\usepackage{mathtools}
\usepackage{amssymb}
\usepackage{bbm}
\usepackage{amsthm}
%\usepackage[tt]{titlepic}% Centre le titre
%\usepackage{fancyhdr}   % Permet de modifier l'entête & footer
\usepackage{caption}     % Permet d'ajouter des légendes en images sans les mettre en float + dans la marge + ref vers le haut de l'envirronement
\usepackage{wrapfig}
\usepackage{fullpage}
%\usepackage{multicol}   % pour les liste sur plusieurs colonnes
%\usepackage{subfigure}  % alligne deux images cote a cote
\usepackage{float}      %permet de mettre du texte entre les figures grace a [H]. Génial! 
\usepackage{eso-pic}    % Fond d'écran page de garde
\usepackage{adjustbox}  % Empêche les box de sortir de la page


%%% Math %%%
%\usepackage{delarray} % Belles matrices
\usepackage{siunitx}


%%% Codes %%%
%\usepackage{listings}
%\usepackage[final]{pdfpages} %% Inclusion fichier pdf

%% Reference
\usepackage{hyperref}
%\renewcommand*{\figureautorefname}{fig.}
%\def\appendixautorefname{annexe}
%\def\tableautorefname{tab.}
%\renewcommand*{\chapterautorefname}{ch.}
%\newcommand{\subfigureautorefname}{\figureautorefname}



%%%%%%%%%%%%%%%%%
%%% Commandes %%%
%%%%%%%%%%%%%%%%%

%%% Physique %%%
\newcommand{\cst}{\text{cst}}
\newcommand{\D}{\partial}
\newcommand{\E}{\vec E}
\newcommand{\B}{\vec B}
\newcommand{\F}{\vec F}
\newcommand{\modu}[1]{|$#1$|}

%%% Math %%%
\newcommand{\oiint}{\int\!\!\!\!\!\!\! \:\!\subset\!\!\supset\!\!\!\!\!\!\!\int}
\newcommand{\rot}{\operatorname{\vec{rot}}}
\newcommand{\divv}{\operatorname{div}}
\newcommand{\phas}[1]{\underline{#1}}
\newcommand{\RE}{\text{Re}}
\newcommand{\ft}{\overset{\mathcal{F}}{\longleftrightarrow}}
\newcommand{\lt}{\overset{\mathcal{L}}{\longleftrightarrow}}
\newcommand{\DS}{\displaystyle}
\newcommand{\Tr}{\operatorname{Tr}}



%% Box
\shorthandon{:}
\newcommand{\theor}[1]{\adjustbox{minipage=\linewidth-2\fboxsep-2\fboxrule,fbox}{\textsc{\iflanguage{british}{Theorem}{Théorème}: }#1}}
\newcommand{\defi}[1]{\adjustbox{minipage=\linewidth-2\fboxsep-2\fboxrule,fbox}{\textsc{\iflanguage{british}{Definition}{Définition}: }#1}}
\newcommand{\lemme}[1]{\adjustbox{minipage=\linewidth-2\fboxsep-2\fboxrule,fbox}{\textsc{\iflanguage{british}{Lemma}{Lemme}: }#1}}
\newcommand{\prop}[1]{\adjustbox{minipage=\linewidth-2\fboxsep-2\fboxrule,fbox}{\textsc{\iflanguage{british}{Property}{Propriété}}\\ #1}}
\newcommand{\proposition}[1]{\adjustbox{minipage=\linewidth-2\fboxsep-2\fboxrule,fbox}{\textsc{Proposition}\\#1}}
\newcommand{\cadre}[1]{\adjustbox{minipage=\linewidth-2\fboxsep-2\fboxrule,fbox}{#1}}
\newcommand{\retenir}[1]{\adjustbox{minipage=\linewidth-2\fboxsep-2\fboxrule,fbox}{\textbf{\textit{\textsc{\iflanguage{british}{To remember}{À retenir}}: }}#1}}

\newcommand{\corollaire}[1]{\bigbreak\begin{tabular}{||c}
	\begin{minipage}{\textwidth}
		\textsc{\iflanguage{british}{Corollary}{Corollaire}: } \textit{#1}
	\end{minipage}
	\end{tabular}}
\newcommand{\exemple}[1]{\bigbreak\begin{tabular}{|c}
	\begin{minipage}{\textwidth}
		\textsc{\iflanguage{british}{Example}{Exemple}: } #1
	\end{minipage}%
	\end{tabular}}%
\shorthandoff{:}
    

%\pagestyle{headings} % Titre du ch et numéro page dans l'entete
%\renewcommand{\proofname}{Démonstration}
%\addto\captionsfrench{\def\tablename{Tableau}}


%%% Background %%%
\newcommand\BackgroundPic{%
	\put(0,0){%
		\parbox[b][\paperheight]{\paperwidth}{%
			\vfill
			\centering
			\includegraphics[width=\paperwidth,height=\paperheight,%
			keepaspectratio]{../../Builder/ulb.jpg}%
			\vfill
}}}

%%% Annexes Cedu %%%
%\usepackage{calrsfs}
%\DeclareMathAlphabet{\pazocal}{OMS}{zplm}{m}{n}
\usepackage{fourier-orns}

\setlength{\parindent}{0pt} 

%%% Attributs %%%
\newcommand*{\NomduCours}[2]{\def\cours{#1}\def\memo{#2}}
\newcommand*{\annee}[2]{\def\adebut{#1}\def\afin{#2}}

\newcounter{auteurcnt}
\newcommand\addauteur[2]{%
	\stepcounter{auteurcnt}%
	\csdef{auteur\theauteurcnt}{\mbox{#1~\textsc{#2}}}}
\newcommand\getauteur[1]{%
	\csuse{auteur#1}}

\newcounter{illustrateurcnt}
\newcommand\addillustrateur[2]{%
	\stepcounter{illustrateurcnt}%
	\csdef{illustrateur\theillustrateurcnt}{\mbox{#1~\textsc{#2}}}}
\newcommand\getillustrateur[1]{%
	\csuse{illustrateur#1}}

\newcounter{rappeltheocnt}
\newcommand\addrappeltheo[2]{%
	\stepcounter{rappeltheocnt}%
	\csdef{rappeltheo\therappeltheocnt}{\mbox{#1~\textsc{#2}}}}
\newcommand\getrappeltheo[1]{%
	\csuse{rappeltheo#1}}

\newcounter{professeurcnt}
\newcommand\addprofesseur[2]{%
	\stepcounter{professeurcnt}%
	\csdef{professeur\theprofesseurcnt}{\mbox{#1~\textsc{#2}}}}
\newcommand\getprofesseur[1]{%
	\csuse{professeur#1}}

\newcounter{iter}